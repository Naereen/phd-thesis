%!TEX root = ../PhD_thesis__Lilian_Besson

\chapter{Expert aggregation for online MAB algorithms selection}
\label{chapter:25}

\graphicspath{{2-Chapters/2-Chapter/Images/}}

\abstractStartChapter{}%
%
The review of MAB algorithms given in the previous Chapter~\ref{chapter:2} showed that there are a large collection of different algorithms designed for different kinds of bandit problems.
In this chapter, we discuss the question of online algorithm selection.
%
We tackle the question of how to select a particular bandit algorithm when a practitioner is facing a particular (unknown) bandit problem.
Instead of always choosing a fixed algorithm, or running costly benchmarks before real-world deployment of the chosen algorithm, another solution could be to select a few candidate algorithms, where at least one is expected to be very efficient for the given problem, and use online algorithm selection to automatically and dynamically decide the best candidate.
We propose an extension of the \ExpFour{} algorithm for this problem, that we called \Aggr, and illustrate its performance on some bandit problems.


\begin{small}
    \begin{flushright}
        % https://fr.wikiquote.org/wiki/Kaamelott/%C3%89lias_de_Kelliwic%E2%80%99h
        \emph{-- Y'a toujours au moins deux solutions à un problème.}\\
        Bruno Fontaine, \emph{Kaamelott}, Livre III, Épisode 67 ``La Potion de Fécondité II''.
    \end{flushright}
\end{small}


\minitocStartChapter{}

% \newpage

% ----------------------------------------------------------------------------
\section{Different approaches on algorithm selection}
\label{sec:25:chooseYourPreferredBanditAlgorithm}

For any real-world applications of MAB algorithms,
several solutions have been explored based on various models, and for any model, typically there are many algorithms available, as we have seen above.
%
We gave in Section~\ref{sec:2:famousMABalgorithms} an overview of the main families of algorithms designed to tackle stochastic MAB problems, and we mainly focused on the $\UCB_1$, \klUCB{} and Thompson sampling algorithms.
% as they are used in the rest of the manuscript.
But a rich research literature has produced many different algorithms tackling the MAB problem, as suggested for instance by the length of the reference book \cite{LattimoreBanditAlgorithmsBook}.
%
Thus, when a practitioner is facing a problem where MAB algorithms could be used, it is not an easy task to decide which algorithm to implement.
It is hard to predict which solution could be the best for real-world conditions at every instant,
and even by assuming a stationary environment, when one is facing a certain problem but has limited information about it, it is hard to know beforehand which algorithm can be the best solution.

In this chapter, we first present two naive approaches for selecting an algorithm when facing a new problem, and then we detail the online approach that uses a ``leader'' MAB algorithm running on top of a pool of ``followers'' algorithms, and we present our contribution that is a new ``leader'' algorithm based on \ExpQ.


\paragraph{Publication.}
%
This chapter is based on our article \cite{Besson2018WCNC}.


\paragraph{Outline.}
%
This chapter is organized as follows.
First, we give more motivations in the rest of this section, then we explain how to combine such stationary MAB algorithms and aggregate them.
We present the proposed algorithm, called \Aggr, in Section~\ref{sub:25:Aggr},
Finally, we present numerical experiments in Section~\ref{sub:25:numExp},
on Bernoulli and non-Bernoulli MAB problems,
comparing the regret of several algorithms against different aggregation algorithms.
Theoretical guarantees are shortly discussed in Section~\ref{sec:25:theory}.
% , before we conclude in Section~\ref{sec:25:conclusion}.


% ----------------------------------------------------------------------
\subsection{Motivation for online algorithm selection}\label{sub:25:introduction}

Many different learning algorithms have been proposed by the machine learning community,
and most of them depend on several parameters, for instance $\alpha$ for $\UCB_1$, the prior for Thompson sampling or BayesUCB,
the $\kl$ function for \klUCB{} etc.
Every time a new MAB algorithm $\Alg$ is introduced, it is compared and benchmarked on some bandit instances, parameterized by $\boldsymbol{\mu} = (\mu_1,\dots,\mu_K)$, usually by focusing on its expected regret $R_T^{\Alg}$.
%
For a known and specific instance, simulations help to select the best algorithm in a pool of algorithms,
but when one wants to tackle an \emph{unknown} real-world problem, one expects to be efficient against \emph{any} problem, of any size and complexity in a certain family:
ideally one would like to use an algorithm that can be applied identically against any problem of such family.


\paragraph{Naive approaches.}
%
On the one hand, a practitioner can decide to pick one algorithm, maybe because it seems efficient on other problems, or maybe because it is simple enough to be used in its application. It might be unrealistic to implement complicated algorithms on limited hardware such as embedded chips in a very low-cost IoT end-device, and for instance a practitioner could chose to only consider the $\UCB_1$ algorithm (or other low-cost algorithms).
%
On the other hand, if prior knowledge on the application at hand is available, one could implement some benchmarks, and compare a set of algorithms on different problems. If a leader appears clearly, it is then possible to choose it for the application.


\paragraph{Illustrative example.}
%
For instance, if you know that the considered problem can either have $K$ arms with very close means, or one optimal arm far away from the other, two versions of $\UCB_1$ will perform quite differently in the two problems:
using a large $\alpha$, \ie, favoring exploration, will give low regret in the first case and high regret in the second case,
while using a low $\alpha$, \ie, favoring exploitation, will obtain opposite performances.
A first approach can be to use an intermediate value, as $\alpha=1/2$ suggested by theory, but another approach could be to consider an aggregated vote of different versions of $\UCB_1$, each running with a different value of $\alpha$ (\eg, in a logarithmic grid), and let a ``leader'' learning algorithm decide which value of $\alpha$ is the best \emph{for the problem at hand}.


% \paragraph{The online approach:}
% %
% Another possibility is to consider an online approach, which is interested in the case where the computation power or memory storage of the application is not a limitation factor, but where one cannot run benchmarks before deploying the application.
% We consider a fixed set of algorithms, and we use another learning algorithm on top of this set, to learn \emph{on the fly} which one should be trusted more, and eventually, used on its own.

% The aggregation approach is especially interesting if we know that the problem the application will face is one of a few known kinds of problems.
% In such cases, if there are $N$ different sorts of problems, and if the practitioner has prior knowledge on it, one can use the naive approach to select an algorithm $\cA_i$ which should perform well on problem $i$, for $i\in[N]$,
% and use the aggregation of $\cA_1,\dots,\cA_N$ when facing the unknown problem.



\subsection{Online algorithm selection with expert aggregation}

The online approach is interesting in the case where the computation power or memory storage of the application is not a limitation factor, but where one cannot run benchmarks before deploying the application.
We consider a fixed set of algorithms, and we use another learning algorithm on top of this set, to learn \emph{on the fly} which one should be trusted more, and eventually, used on its own.

% - ``Aggregation of Multi-Armed Bandits learning algorithms for Opportunistic Spectrum Access'', see https://hal.inria.fr/hal-01705292

For any real-world applications of MAB algorithms, several solutions have been explored based on various models.
However, it is hard to exactly predict which solution could be the best for real-world conditions at every instants.
As we saw before, a possible approach is to pick one algorithm and simply use it, or to run simulations before-hand, if possible.
But a third possible approach is to select \emph{on the run} the best algorithm for a specific situation, for which the so-called \emph{expert aggregation algorithms} can be useful.
%
To the best of our knowledge, aggregation algorithms, such as \ExpQ{} dating back from 2002 (\cite{Auer02}),
have never been used for stochastic MAB problems,
and we show that it appears empirically sub-efficient when applied to simple stochastic problems.
%
In this Section, we present an improved variant of \ExpQ, called \Aggr.
For synthetic MAB problems,
with Bernoulli distributions or other distributions for arms,
simulation results are presented to demonstrate its empirical efficiency.
We combine classical algorithms, such as Thompson sampling, Upper-Confidence Bounds algorithms (\UCB{} and variants), and Bayesian or Kullback-Leibler UCB.
%
Our algorithm offers good performance compared to state-of-the-art algorithms
(\ExpQ{}, \CORRAL{} or \LearnExp{}) \cite{Agarwal16,Singla17}),
and appears as a robust approach to select on the run the best algorithm for any stochastic MAB problem, being more realistic to real-world radio settings than any tuning-based approach.

\TODOL{This chapter is basically a raw include from my paper ``Aggregation of Multi-Armed Bandits learning algorithms for Opportunistic Spectrum Access'', see https://hal.inria.fr/hal-01705292}


% ----------------------------------------------------------------------
\subsection{Motivation for online algorithm selection}\label{sub:25:introduction}

Many different learning algorithms have been proposed by the machine learning community,
and most of them depend on several parameters, for instance $\alpha$ for \UCB, the prior for Thompson sampling or BayesUCB,
the $\kl$ function for \klUCB{} etc.
Every time a new MAB algorithm $\Alg$ is introduced, it is compared and benchmarked on some bandit instance, parameterized by $\boldsymbol{\mu} = (\mu_1,\dots,\mu_K)$, usually by focusing on its expected regret $R_T = \E_{\boldsymbol{\mu}}[\widetilde{R_T}]$.
%
For a known and specific instance, simulations help to select the best algorithm in a pool of algorithms.
But when one wants to tackle an \emph{unknown} real-world problem, one expects to be efficient against \emph{any} problem, of any kind, size and complexity:
ideally one would like to use an algorithm that can be applied identically against any problem.
To choose the best algorithm, two approaches can be followed:
either extensive benchmarks are done beforehand -- if this is possible -- to select the algorithm and its optimal parameters, or an adaptive algorithm is used to learn \emph{on the fly} its parameters.
We present a simple adaptive solution, that aggregates several learning algorithms in parallel and adaptively chooses which one to trust the most.


\paragraph{Outline}
This Section is organized as follows.
First, we explain in Section~\ref{sub:25:aggregation} how to combine such algorithms for aggregation.
Then we present our proposed algorithm, called \Aggr, in Section~\ref{sub:25:Aggr},
Finally, we present numerical experiments in Section~\ref{sub:25:numExp},
on Bernoulli and non-Bernoulli MAB problems,
comparing the regret of several algorithms against different aggregation algorithms.
Theoretical guarantees are shortly discussed in Section~\ref{sub:25:theory}, and Section~\ref{sub:25:conclusion} concludes.


% ----------------------------------------------------------------------
\subsection{Aggregating bandit algorithms}\label{sub:25:aggregation}

We assume to have $N \geq 2$ MAB algorithms, $\Alg_1, \dots, \Alg_N$,
and let $\Alg_{\mathrm{aggr}}$ be an aggregation algorithm,
which runs the $N$ algorithms in parallel (with the same slotted time), and use them to choose its channels based on a voting from their $N$ decisions.
%
$\Alg_{\mathrm{aggr}}$ depends on a pool of algorithms and a set of parameters.
We would like that $\Alg_{\mathrm{aggr}}$
performs almost as well as the best of the $\Alg_a$, with a good choice of its parameters, independently of the MAB problem.
Ideally $\Alg_{\mathrm{aggr}}$ should perform similarly to the best of the $\Alg_a$.
%
To simplify the presentation, we only aggregate bandit algorithms that give deterministic recommendations:
one arm is chosen with probability $1$ and the others with probability $0$.
However, both \ExpQ{} and \Aggr{} can be adapted to aggregate randomized bandit algorithms, \ie, algorithms that output a probability distribution $\xi_t$ over the arms $\{1,\dots,K\}$ at each time step, and draw the next selected arm according to this distribution.

The aggregation algorithm maintains a probability distribution $\pi^{t}$ on the $N$ algorithms $\Alg_a$, starting from a uniform distribution:
$\pi^t_a$ is the probability of trusting the decision made by algorithm $\Alg_a$ at time $t$.
$\Alg_{\mathrm{aggr}}$ then simply performs a weighted vote on its algorithms: it decides whom to trust by sampling $a \in \{1,\dots,N\}$ from $\pi^t$, then follows $\Alg_a$'s decision.
The main questions are then to know what observations (\ie, arms and rewards) should be given as feedback to which algorithms,
and how to update the trusts at each step, and our proposal \Aggr{} differs from \ExpQ{} on these very points.


% ----------------------------------------------------------------------
\subsubsection{The \Aggr{} algorithm}\label{sub:25:Aggr}

Our proposed \Aggr{} is detailed in Algorithm~\ref{algo:25:Aggr}.
%
\begin{small}
	\begin{figure}[h!]
		\centering
		% Documentation at http://mirror.ctan.org/tex-archive/macros/latex/contrib/algorithm2e/doc/algorithm2e.pdf if needed
		% Or https://en.wikibooks.org/wiki/LaTeX/Algorithms#Typesetting_using_the_algorithm2e_package
		\begin{algorithm}[H]
			% XXX Input, data and output
			\KwIn{$N$ bandit algorithms, $\Alg_1, \dots, \Alg_N$, with $N \geq 2$}
			\KwIn{Number of arms, $K \geq 2$}
			\KwIn{Time horizon, $T \geq 1$, \textbf{not} used for the learning}
			\KwIn{A sequence of learning rates, $(\eta_t)_{t \geq 1}$}
			\KwData{Initial uniform distribution, $\pi^{0} = U(\{1, \dots, N\})$}
			\KwResult{$\Alg_{\mathrm{aggr}}=\Aggr\left[\Alg_1, \dots, \Alg_N\right]$}
			% XXX Algorithm
			\For(\tcp*[f]{At every time step})
            {$t = 1, \dots, T$}{
				%
				% \tcp{First, run each algorithm $\Alg_a$}
				\For(\tcp*[f]{Can be parallel}){$a = 1, \dots, N$}{
					$\Alg_a$ updates its indexes (e.g., \UCB{} indexes)\;
					It chooses $A^{t+1}_{a} \in \{1, \dots, K\}$.
				}
				%
				% \tcp{The more algorithms have chosen arm $j$, the more probable it is to be chosen.}
				Let $p^{t+1}_j := \sum\limits_{a = 1}^{N} \pi^{t}_a \times \mathbbm{1}(\{A^{t+1}_{a} = j\}), \;\forall 1\leq j \leq K$\;
				% \tcp*{Sum trusts}
				Then $\Alg_{\mathrm{aggr}}$ chooses arm $A^{t+1} \sim p^{t+1}$\;
				% $\P(A^{t+1}_{\mathrm{aggr}} = j) = p^{t+1}_j$
				% \tcp*{Sample once}
				%
				Give \emph{original} reward $(A^{t+1}, 1 - \ell_{A^{t+1}}(t+1))$ to \emph{each} $\Alg_a$ (maybe \emph{not} on its chosen arm)\;
				Compute an \emph{unbiased} estimate of the loss of the trusted algorithms,$$ \ell^{t+1} = \ell_{A^{t+1}}(t+1) / p^{t+1}_{A^{t+1}}\;. $$
				% \tcp{Use bandit feedback to update $\pi^{t}$}
				% \tcp{Finally, use the instantaneous bandit feedback (loss $\ell_{A^{t+1},t}$) to update $\pi^{t}$}
				% \For(\tcp*[f]{Can be parallel}){$a = 1, \dots, N$}{
				\For{$a = 1, \dots, N$}{
					% \eIf{$\Alg_a$ was trusted, \ie, $A^{t+1}_{a} = A^{t+1}$}{
					\If{$\Alg_a$ was trusted, \ie, $A^{t+1}_{a} = A^{t+1}$}{
						$ \pi^{t+1}_{a} = \exp(\eta_t \ell^{t+1}) \times \pi^{t}_{a} $
						% \tcp*{More trusted}
						}%{
						% $ \pi^{t+1}_{a} = \exp(- \eta_t \ell^{t+1}) \times \pi^{t}_{a} $
						% \tcp*{Less trusted}
					% }
				}
				Renormalize the new $\pi$: $\pi^{t+1} := \pi^{t+1} / \sum_{a=1}^{N} \pi^{t+1}_{a}$.
				% \tcp*{Project it back to $\Delta_N$}
			}
			\caption{Our aggregation algorithm \Aggr.}
			\label{algo:25:Aggr}
		\end{algorithm}
	\end{figure}
\end{small}


At every time step, after having observed a loss $\ell_{A^{t+1}}(t+1)$ for its chosen action $A^{t+1}$,
the algorithm updates the trust probabilities from $\pi^t$ to $\pi^{t+1}$ by
a multiplicative exponential factor (using the learning rate and the \emph{unbiased} loss).
%
Only the algorithms $\Alg_a$ who advised the last decision get their trust updated, in order to trust more the ``reliable'' algorithms.
The loss estimate is unbiased in the following sense. If one had access to the rewards $r_k(t+1)$ (or the losses $\ell_k(t+1)$) for all arms $k$, the loss incurred by algorithm $a$ at time $t+1$ would be $\tilde{\ell}_{a,t+1} = \ell_{A_a^{t+1}}(t+1)$. This quantity can only be observed for those algorithms for which $A_{a}^{t+1}=A^{t+1}$. However, by dividing by the probability of observing this recommendation, one obtains an unbiased estimate of $\tilde{\ell}_{a,t}$. More precisely,
\[\hat{\ell}_{a,t+1} := \frac{\ell_{A_a^{t+1}}(t+1)}{p_{A_a^{t+1}}^{t+1}} \mathbbm{1}(A_a^{t+1} = A^{t+1})\] satisfies $\mathbb{E}[\hat{\ell}_{a,t+1} | \mathcal{H}_t] = \tilde{\ell}_{a,t+1}$, for all $a$, where the expectation is taken conditionally to the history of observations up to round $t$, $\mathcal{H}_{t}$. Observe that $\tilde{\ell}_{a,t+1}=\ell^{t+1}$ for all algorithms $a$ such that $A_a^{t+1}=A^{t+1}$, and $\tilde{\ell}_{a,t+1}=0$ otherwise.


% - Precise some tricky points
An important feature of \Aggr{} is the feedback provided to each underlying bandit algorithm, upon the observation of arm $A^{t+1}$. Rather than updating only the trusted algorithms (that is the algorithms which would have drawn arm $A^{t+1}$) with the observed reward $r_{A^{t+1}}(t+1)=1 - \ell_{A^{t+1}}(t+1)$,  we found that updating each algorithm with the (original) loss observed for arm $A^{t+1}$ improves the performance drastically.
%
As expected, the more feedback they get, the faster the underlying algorithms learn, and the better the aggregation algorithm will be. This intuition is backed up by theory explained in \cite{Maillard11}.


Regarding the update of $\pi^t$, one can note that the trust probabilities are not all updated before the normalization step,
and an alternative would be to
increase $\pi_{a}$ if $A^{t+1}_a = A^{t+1}$ and to decrease it otherwise.
It would not be so different, as there is a final renormalization step, and empirically this variation has little impact on the performance of \Aggr{}.



% ----------------------------------------------------------------------
\subsubsection{Explanations of \Aggr{} versus \ExpQ{} }\label{sub:25:Exp4}

% - explain, but don't give pseudo-code ?
The \ExpQ{} algorithm (see, e.g. \cite[Section 4.2]{Bubeck12})
is similar to \Aggr{}, presented in Algorithm \ref{algo:25:Aggr},
but differs in the two following points.
%
First, $a\sim\pi^t$ is sampled first and the arm chosen by $\Alg_a$ is trusted, whereas \Aggr{}
needs to listen to the $N$ decisions to perform the updates on $\pi^{t+1}$,
and \ExpQ{} gives back an observation (arm, reward) only to the last trusted algorithm
whereas \Aggr{} gives it to all algorithms.
%
Second, after having computed the loss estimate $\ell$, \ExpQ{} updates the estimated cumulative loss for each algorithm,
$\widetilde{L}_a(t) = \sum_{s=1}^{t} \ell_{A_a^s}(s) \times \mathbbm{1}(A^{s}_{a} = A^{s}_{\mathrm{aggr}})$.
%
Instead
\footnote{It is the same in the case of constant learning rates, and does not differ much even for decreasing learning rates.}
of updating $\pi^{t}$ multiplicatively as we do for our proposal, \ExpQ{} recomputes it, proportionally to
$\exp(- \eta_t \widetilde{L}_a(t))$.

\paragraph{How to choose the learning rates $(\eta_t)$?}
%
The sequence of non-negative learning rates $(\eta_t)_{t \geq 1}$ used by \ExpQ{} can be arbitrary, it can be constant but should be non-increasing \cite[Theorem 4.2]{Bubeck12}.
% helps to choose a good sequence.
If the horizon $T$ is known (and fixed), the best choice is given by $\eta_t = \eta := 2 \log(N) / (T K)$.
However, for real-world communication problems, it is unrealistic to assume a fixed and known time horizon, so we prefer the alternative horizon-free choice%
\footnote{~It is non-increasing, and is obtained by minimizing the upper-bound on the contextual regret derived in \cite[pp48]{Bubeck12}.}
of learning rates,
$\eta_t := \log(N) / (t K)$ suggested by \cite{Bubeck12}.
We compare both approaches empirically, and the second one usually performs better.
We also stick to this choice of $(\eta_t)_{t \geq 1}$ for \Aggr.



% ----------------------------------------------------------------------
\subsection{Experiments on simulated MAB problems}\label{sub:25:numExp}

% - explain the problem studied in the simulation
We focus on \emph{i.i.d.} MAB problems, with $K = 9$ channels\footnote{~Similar behaviors are observed for any not-too-large values of $K$, we tried up-to $K = 100$ and the same results were obtained.}.
For Bernoulli problem, the first one uses $\boldsymbol{\mu}=[0.1,\dots,0.9]$,
and the second one is divided in three groups:
2 very bad arms ($\mu = 0.01, 0.02$), 5 average arms ($\mu = 0.3$ to $0.6$) and 3 very good arms ($\mu = 0.78, 0.8, 0.82$).
The horizon is set to $T = 20000$ (but its value is unknown to all algorithms), and simulations are repeated $1000$ times, to estimate the expected regret.
%
This empirical estimation of the expected regret $R_T$ is plotted below, as a function of $T$, comparing some algorithms $\Alg_1,\dots,\Alg_N$ (for $N=6$), and their aggregation with \Aggr{} (displayed in orange bold),
using the parameter-free learning rate sequence, $\eta_t := \log{N} / (t K)$.
The Lai \& Robbins' logarithmic lower-bound \cite{LaiRobbins85} is also plotted, and it is crucial to note that it is only asymptotic and to not be surprised by having regret curves smaller than the lower-bound (\eg, for the easier Bernoulli problem).
%
Note that for each of the $1000$ simulations, we choose to generate all the rewards beforehand, \ie, one full matrix $(r_k(t))_{1\leq k \leq K, 1 \leq t \leq T}$ for every repetition, in order to compare the algorithms on the same realizations%
\footnote{Similar plots and similar results are obtained if this trick is not used, but it makes more sense to compare them against the same randomization.}
of the MAB problem.

We compare our \Aggr{} algorithm,
as well as other aggregation algorithms, \ExpQ{},
\CORRAL{} and \LearnExp{} (both with default parameters) \cite{Bubeck12,Agarwal16,Singla17}.
The aggregated algorithms consist in a naive uniform exploration (to have at least one algorithm with bad performances, \ie{} linear regret, but it is not included in the plots),
\UCB{} with $\alpha=1/2$, three \klUCB{} with Bernoulli, Gaussian and exponential $\kl$ functions, and BayesUCB and Thompson sampling with uniform prior.

Figures~\ref{fig:25:EasyBernoulli} and \ref{fig:25:HarderMixed} are in semilog-$y$ scale, this helps to see that the best algorithms can be an \emph{order of magnitude} more efficient than the worst, and the \Aggr{} performs similarly to the best ones, when the other aggregation algorithms are usually amongst the worst.
Figure~\ref{fig:25:HarderMixed_semilogx} is in semilog-$x$ scale to show that the regret of efficient algorithms are indeed logarithmic.


\begin{figure}[h!]  % [htbp]
	\centering
	\includegraphics[width=1.00\linewidth]{2-Chapters/2-Chapter/IEEE_WCNC_2018.git/plots/main_semilogy____env1-4_932221613383548446.pdf}
	\caption{On a ``simple'' Bernoulli problem (semilog-$y$ scale).}
	\label{fig:25:EasyBernoulli}
\end{figure}

\begin{figure}[b!]  % [htbp]
	\centering
	\includegraphics[width=1.00\linewidth]{2-Chapters/2-Chapter/IEEE_WCNC_2018.git/plots/main____env2-4_932221613383548446.pdf}
	\caption{On a ``harder'' Bernoulli problem, they all have similar performances, except \LearnExp.}
	\label{fig:25:HardBernoulli}
\end{figure}

\begin{figure}[b!]  % [htbp]
	\centering
	\includegraphics[width=1.00\linewidth]{2-Chapters/2-Chapter/IEEE_WCNC_2018.git/plots/main____env3-4_932221613383548446.pdf}
	\caption{On an ``easy'' Gaussian problem, only \Aggr{} shows reasonable performances, thanks to BayesUCB and Thompson sampling.}
	\label{fig:25:EasyGaussian}
\end{figure}

\begin{figure}[h!]  % [htbp]
	\centering
	\includegraphics[width=1.00\linewidth]{2-Chapters/2-Chapter/IEEE_WCNC_2018.git/plots/main_semilogy____env4-4_932221613383548446.pdf}
	\caption{On a harder problem, mixing Bernoulli, Gaussian, Exponential arms, with 3 arms of each types with the \emph{same mean}.}
	\label{fig:25:HarderMixed}
\end{figure}

\begin{figure}[b!]  % [htbp]
	\centering
	\includegraphics[width=1.00\linewidth]{2-Chapters/2-Chapter/IEEE_WCNC_2018.git/plots/main_semilogx____env4-4_932221613383548446.pdf}
	\caption{The semilog-$x$ scale clearly shows the logarithmic growth of the regret for the best algorithms and our proposal \Aggr, even in a hard ``mixed'' problem (\emph{cf}. Figure~\ref{fig:25:HarderMixed}).}
	\label{fig:25:HarderMixed_semilogx}
\end{figure}

For Bernoulli problems (Figures~\ref{fig:25:EasyBernoulli} and \ref{fig:25:HardBernoulli}), UCB with $\alpha=1/2$, Thompson sampling, BayesUCB and \klUCB{}$^+$ (with the binary $\kl$ function) all perform similarly, and \Aggr{} is found to be as efficient as all of them.
For Gaussian and exponential arms, rewards are truncated into $[0,1]$, and the variance of Gaussian distributions is fixed to $\sigma^2 = 0.05$ for all arms, and can be known to the algorithms (the \kl{} function is adapted to this one-dimensional exponential family).
%
Figure~\ref{fig:25:EasyGaussian} uses only Gaussian arms, with a large gap between their means and a relatively small variance, giving an ``easy'' problem.
%
And Figure~\ref{fig:25:HarderMixed} shows a considerably harder ``mixed'' problem, when the distributions are no longer in the same one-dimensional exponential family and so the Lai \& Robbins' lower-bound no longer holds (even if there still exists a lower-bound).

For each of the 4 problems considered, the \Aggr{} algorithm with default option (broadcast loss to all players) is the best of all the aggregation algorithms,
and its regret is very close to the best of the aggregated algorithms.
Especially in difficult problems with mixed or unknown distributions, \Aggr{} showed to be more efficient that \ExpQ{} and orders of magnitude better than the other reference aggregation algorithms \LearnExp{} and \CORRAL{} (see Figures~\ref{fig:25:HarderMixed} and \ref{fig:25:HarderMixed_semilogx}).


% ----------------------------------------------------------------------
\subsection{No theoretical guarantees yet}\label{sub:25:theory}

The \Aggr{} does not have satisfying theoretical guarantees in terms of regret $R_T$ yet, unlike many bandit algorithms.
%
Another notion, the \emph{adversarial regret}, denoted by $\overline{R_T}$,
measures the difference in terms of rewards,
between the aggregation algorithm $\Alg_{\mathrm{aggr}}$ and the best aggregated algorithm $\Alg_a$. This is in contrast with the (classical) regret, which measure the difference with the best fixed-arm strategy (Eq.~\eqref{eq:5:classicalregret}). Thus,  even if the aggregated algorithms have logarithmic (classical) regret, having an adversarial regret scaling as $\sqrt{T}$ does not permit to exhibit a logarithmic (classical) regret for the aggregation algorithm.
%
%
Under some additional hypotheses,
\cite[Theorem 4.2]{Bubeck12} proves that
\ExpQ{} satisfies a bound on adversarial regret, % $\overline{R_T}$
$\overline{R_T} \leq 2 \sqrt{T N \log{k}}$,
with the good choice of the learning rate sequence $(\eta_t)_{t \geq 1}$.
Our proposed algorithm follows quite closely the architecture of \ExpQ,
and a similar bound for \Aggr{} is expected to hold.
% this is left as a future work.
% For sake of conciseness, we cannot present a proof here.
%

This would be a first theoretical guarantee, but not satisfactory as simple algorithms like UCB have regrets scaling as $\log{T}$ \cite{Auer02,Bubeck12}, not $\sqrt{T}$.
%
Regret bounds in several different settings are proved for the \CORRAL{} algorithm \cite{Agarwal16}, but no logarithmic upper-bound can be obtained from their technique, even in the simplest setting of stochastic bandits.
%
However, \Aggr{} always seems to have a (finite-horizon) logarithmic regret in all the experiments we performed,
for both Bernoulli and non-Bernoulli problems (\eg, Gaussian, exponential and Poisson distributions).
Further theoretical developments are left as future work.


% ----------------------------------------------------------------------
\subsection{Conclusion about our \Aggr{} algorithm}\label{sub:25:conclusion}

We presented the use of aggregation algorithms in the context of Opportunistic Spectrum Access for Cognitive Radio,
especially for the real-world setting of unknown problem instances,
when tuning parameters before-hand is no longer possible and an adaptive algorithm is preferable.
% - \ExpQ and \Aggr works fine
% Both algorithms \ExpQ{} and \Aggr{} have been detailed, and we explained why
Our proposed \Aggr{} was presented in details,
and we also highlighted its differences with \ExpQ.
% as well \color{red} as the intuition that it seems more suited for purely stochastic problems\color{black}.

% - \Aggr{} really help to select the best algorithm against a certain problem, on the fly without any prior knowledge of the problem neither any prior knowledge on which algorithms is the best
We realized experiments on simple MAB problems already used in the community of bandit algorithms for OSA \cite{Jouini10},
and the simulations results showed that \Aggr{} works as expected, being able to identify on the fly the best algorithm to trust for a specific problem.
Experiments on problems mixing different families of distributions were also presented, with similar conclusions in favor of \Aggr.
It is not presented in this Section, but our proposed algorithm also works well in dynamic scenarios, in which the distribution of the arms can change abruptly at some time,
and appears to be more robust than simple non-aggregated algorithms.

\ExpQ{} has theoretical guarantees in terms of adversarial regret, and even if the same result could hold for \Aggr, results in terms of classical regret are yet to be proved.
Empirically, \Aggr{} showed to always have a logarithmic
regret $R_T$ if it aggregates algorithms with logarithmic regrets (like UCB, \klUCB, Thompson sampling, BayesUCB etc).
It usually succeeds to be close to the best of the aggregated algorithms, both in term of regret and best arm pull frequency.
As expected, the \Aggr{} is never able to outperform any of the aggregated algorithms, but this was an over-optimistic goal.
%
What matters the most is that, empirically, \Aggr{} is able to quickly discover \emph{on the fly} the best algorithms to trust, and then performs almost as well as if it was following it from the beginning.

Our \Aggr{} algorithm can probably be rewritten as an Online Mirror Descent, as \ExpQ{} and \CORRAL,
but this does not appear useful as in the case of \CORRAL{}  the analysis cannot bring a logarithmic bound on the regret even by aggregating asymptotically optimal algorithms.
We will continue investigating regret bounds for \Aggr,
and other directions include possible applications to the non-stochastic case (\eg, rested or restless Markovian problems, like it was very recently studied in \cite{Luo17}).

\paragraph{A note on the simulation code}
The page \texttt{SMPyBandits.GitHub.io/Aggregation.html} explains how to reproduce the experiments used for this Section.


% ----------------------------------------------------------------------------
% \bibliographystyle{ieeetr}
% \bibliography{IEEE_WCNC__2018__Paper__Lilian_Besson__07-17}


% % A last solution is online algorithm selection, inspired from expert aggregation.
% % Include here the discussion about expert aggregation and my \textbf{Aggregator} algorithm, see https://hal.inria.fr/hal-01705292


% \newpage  % WARNING ?

% % ----------------------------------------------------------------------------
% \section{Conclusion}
% \label{sec:25:conclusion}

% In this chapter, we presented the first contribution of this manuscript,
% that corresponds to our article \cite{Besson2018WCNC}.
% %
% We presented the use of aggregation algorithms,
% especially for the real-world setting of unknown problem instances,
% when tuning parameters before-hand is no longer possible and an adaptive algorithm is preferable.
% % - \ExpQ and \Aggr works fine
% % Both algorithms \ExpQ{} and \Aggr{} have been detailed, and we explained why
% Our proposed \Aggr{} was presented in details,
% and we also highlighted its differences with \ExpQ.

% % as well \color{red} as the intuition that it seems more suited for purely stochastic problems\color{black}.
% %
% % - \Aggr{} really help to select the best algorithm against a certain problem, on the fly without any prior knowledge of the problem neither any prior knowledge on which algorithms is the best
% We presented experiments on simple MAB problems, coming from previous work on bandits for OSA \cite{Jouini10},
% and the simulations results showed that \Aggr{} is able to identify on the fly the best algorithm to trust for a specific problem, as expected.
% Experiments on problems mixing different families of distributions were also presented, with similar conclusions in favor of \Aggr.
% % It is not presented in this section, but our proposed algorithm also works well in dynamic scenarios, in which the distribution of the arms can change abruptly at some time,
% % and appears to be more robust than simple non-aggregated algorithms.

% \ExpQ{} has theoretical guarantees in terms of adversarial regret, and even if the same result could hold for \Aggr, results in terms of classical regret are yet to be proven.
% Empirically, \Aggr{} showed to always have a logarithmic
% regret if it aggregates at least one algorithm with logarithmic regrets (like UCB, \klUCB, Thompson sampling, BayesUCB etc).
% It usually succeeds to be close to the best of the aggregated algorithms, both in term of regret and best arm pull frequency.
% As expected, the \Aggr{} is never able to outperform any of the aggregated algorithms, but this is an over-optimistic goal.

% What matters the most is that, empirically, \Aggr{} is able to quickly discover \emph{on the fly} the best algorithms to trust, and then performs almost as well as if it was following it from the beginning.
% %
% %
% Our \Aggr{} algorithm could probably be rewritten as an Online Mirror Descent (for more details, see for instance \cite{Hazan2016introduction,Zimmert2018}), as it is the case for both \ExpQ{} and \CORRAL.
% But this direction does not appear very useful for our objective at hand, as in the case of \CORRAL{}  the analysis cannot bring a logarithmic regret bound, even by aggregating asymptotically optimal algorithms.
% % We will continue investigating regret bounds for \Aggr,
% % and other directions include possible applications to the non-stochastic case (\eg, rested or restless Markovian problems, like it was very recently studied in \cite{Luo18}).

% % The numerical simulations used our Python open-source library, SMPyBandits, that is presented in details in the next Chapter~\ref{chapter:3}.
% % The next chapter also includes simulations of the most important MAB algorithms that we presented above, on Bernoulli distributed problems of various sizes and durations.
