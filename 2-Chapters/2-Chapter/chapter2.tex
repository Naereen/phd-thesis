%!TEX root = ../thesis.tex

% First chapter begins here
\chapter{Single Player and Stationary Multi-Armed Bandit Models}
\label{chapter:2}
\minitoc
\newpage
% Write miniTOC just after the title
\graphicspath{{2-Chapters/2-Chapter/Images/}}

% ----------------------------------------------------------------------------
\section{Stationary MAB models}
\label{sec:2:notations}

% ----------------------------------------------------------------------------
\subsection{Finite arms and stochastic rewards}

In this thesis, we only consider models with finite arms.
No linear bandits, no contextual bandits etc.

For now, we only consider stochastic rewards. Piece-wise stochastic models are studied in Chapter~\ref{chapter:6}.


% ----------------------------------------------------------------------------
\subsection{Different hypotheses on rewards distributions}

Everything is implemented and documented at
https://smpybandits.github.io/docs/Arms.html

Bernoulli, Gaussian, sub-Bernoulli, sub-Gaussian, Exponential, sub-Exponential...

One dimensional exponential family...

Just bounded, in $[a,b]$ but without loss of generality in $[0,1]$ etc.

Markov models: presented later in Section~\ref{sec:3:markovModels}.


% ----------------------------------------------------------------------------
\section{Applications of stationary MAB}
\label{sec:2:applicationsofStationaryMAB}

\subsection{Various possible applications}

- clinical trial
- A/B testing of websites
- online content recommandation
- wireless communication, lots of possibilities, one is frequency/channel selection in the OSA paradigm, detailed after
- hyper-parameters tuning of deep neural networks etc


\subsection{Applications for Opportunistic Spectrum Access}

Detail the analogy arm = channel in a licensed spectrum, rewards $r_k(t)$ = 1 if no Primary User or 0 if channel $k$ is busy at time $t$.

Detail the previous work from our team SCEE on bandits + OSA : Wassim, Navik


% ----------------------------------------------------------------------------
\section{A short review of famous MAB algorithms}
\label{sec:2:famousMABalgorithms}

Everything is implemented and documented at
https://smpybandits.github.io/docs/Policies.html


\subsubsection{Why do we need smart algorithms?}
Why Follow-the-Leader (EmpiricalMeans https://smpybandits.github.io/docs/Policies.EmpiricalMeans.html) don't work, example.

\subsection{Simple but efficient algorithms when well tuned: $\varepsilon$-greedy and Explore-then-Commit}

See
https://smpybandits.github.io/docs/Policies.EpsilonGreedy.html
https://smpybandits.github.io/docs/Policies.ExploreThenCommit.html


% ----------------------------------------------------------------------------
\subsection{Index policies: UCB and others}

\subsubsection{UCB}
Optimism in face of uncertainty, for UCB etc.


\subsubsection{klUCB}
Optimal upper confidence bounds based on Kullback-Leibler divergence and klUCB indexes

\subsubsection{Other policies inspired by UCB}

- UCB-H and klUCB-H
- UCB-V
- UCB-dagger
- UCBoost
- UCB+ and klUCB+


% ----------------------------------------------------------------------------
\subsection{Bayesian policies: Thompson Sampling and others}

Idea behind Thompson sampling and Bayesian philosophy.

The Beta posterior and others (https://smpybandits.github.io/docs/Policies.Posterior.html) and Thompson Sampling.


% ----------------------------------------------------------------------------
\subsection{Other policies: BESA and others}

It's a good opportunity to show off and say that I worked on my own on some non-standard policies.

- One of the first paper I read in 2016 was defining AdBandits, https://smpybandits.github.io/docs/Policies.AdBandits.html, a mixture between Thompson sampling and BayesUCB (very efficient in practice)
- I especially love the BESA algorithm by [? and ? and Maillard].


% ----------------------------------------------------------------------------
\subsection{Policies for adversarial bandits: Exp3 and others}

Exp3, Exp3++ etc
https://smpybandits.github.io/docs/Policies.Exp3.html
https://smpybandits.github.io/docs/Policies.Exp3PlusPlus.html

Softmax: is it the same, but written differently?
https://smpybandits.github.io/docs/Policies.Softmax.html


% ----------------------------------------------------------------------------
\subsection{Hybrid policies: MOSS, klUCB++ and klUCB-Switch and others}

Recent work.
https://smpybandits.github.io/docs/Policies.MOSS.html
https://smpybandits.github.io/docs/Policies.IMED.html
https://smpybandits.github.io/docs/Policies.klUCBPlusPlus.html
https://smpybandits.github.io/docs/Policies.klUCBSwitch.html

Talk about these?
https://smpybandits.github.io/docs/Policies.OCUCB.html
https://smpybandits.github.io/docs/Policies.OSSB.html

With perturbation: maybe we don't need optimism in face of uncertainty, we just need uniform sampling!
Randomized Confidence Bound (RCB) with uniform perturbations performs as well as UCB!
https://smpybandits.github.io/docs/Policies.RandomizedIndexPolicy.html


\subsection{Summary of classification of the most famous algorithms}

A huge table in one page in portrait mode, giving:
- name
- reference, year
- family, classification
- time/memory complexity ?
- regret bound for stochastic regret

of about 20/30 algorithms?


% ----------------------------------------------------------------------------
\section{Lower and upper bounds on regret}
\label{sec:2:lowerUpperBoundsRegret}


\subsection{Measuring performance with the (mean) regret}
Explain what is regret


\subsection{Regret lower bounds}

- [Lai and Robbins] lower-bound in $\Omega(\log(T))$
- Worst case lower in $\Omega(\sqrt{T})$


\subsection{Regret upper bounds, tools and different kinds of bounds}

- useful decomposition, with number of samples (see Claire's PhD)
- different kinds of bounds, problem-dependent with $\bigO{C \log(T)}$ with $C$ a measure of the problem complexity, or problem-independent with $\bigO{\sqrt{KT}}$ ?
- tools to show regret upper-bounds, concentration inequalities,
- short proof of UCB or klUCB? just to start to show the tools


\subsection{Other measures of performances}

- Quantile regret or just histogram of regrets
- Worst case regret (max $R_T^{r}$ for $r$ index of Monte-Carlo simulation?)
- Fairness


% ----------------------------------------------------------------------------
\section{Different approaches on algorithm selection, or ``How to choose your preferred bandit algorithm?''}
\label{sec:2:chooseYourPreferredBanditAlgorithm}

\subsection{Naive approach: use UCB or Thompson sampling, and forget about the rest!}
First solution: be naive, only use UCB everywhere

\subsection{Use prior knowledge about the future application to select beforehand the chosen algorithm}
Second solution, if one has some prior knowledge about the domain or setting for which the learning algorithms will be deployed, she can run synthetic or real-world simulations, compare many algorithms before deployment, and select the best algorithm!

\subsection{Online algorithm selection}

\subsubsection{}
A last solution is online algorithm selection, inspired from expert aggregation.
Include here the discussion about expert aggregation and my \textbf{Aggregator} algorithm, see https://hal.inria.fr/hal-01705292



% ----------------------------------------------------------------------------
\section{Conclusion of Chapter 2}
\label{sec:2:conclusion}

In this chapter, we saw...

Future works include...



% ----------------------------------------------------------------------------
\section{Appendix for Chapter 2}
\label{sec:2:appendix}


\subsection{Numerical computation of Kullback-Leibler divergences and klUCB indexes}

I want to detail the maths behind an implementation of klUCB, see https://smpybandits.github.io/docs/Policies.kullback.html and all what is detailed there.

How to write a fast version of klBern and klucbBern (the two most used functions), using C or Numba or Cython, maybe I can include the speed up simulations here? First piece of code, before really presenting SMPyBandits ?

