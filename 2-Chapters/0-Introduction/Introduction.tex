%!TEX root = ../thesis.tex

% First chapter begins here
%\adjustmtc
\chapter*{Introduction}
\addcontentsline{toc}{chapter}{Introduction}
\chaptermark{Introduction}
\minitoc
\label{chapter:Intro}
\graphicspath{{2-Chapters/1-Chapter/Images/}}
% Write miniTOC just after the title
%\startcontents[chapters]\vbox{\bf\Large Sommaire \hrule}
%\printcontents[chapters]{}{1}{}\vspace*{1pc}\hrule}
\section*{General context}
\indent The work presented in this PhD thesis applies in aeronautical telemetry, which is the process of remotely collecting measurements done by electronic equipment on board the aircraft and transmitting these data to a distant location for monitoring, display, and recording \cite{telemetry_def}. 

An aeronautical telemetry system is used during the flight testing phase to monitor the behavior of the plane by transmitting the aircraft dynamics settings (such as velocities, vibrations, temperatures, stress and strain parameters,...) from the aircraft to the ground station over a radio-frequency (RF) link. Aeronautical telemetry is a one-way link from the plane to the ground (i.e., there is no uplink mode) that requires real-time data transmission and analysis to ensure the pilot's safety. This makes the use of the aeronautical telemetry a critical task that demands highly reliable systems. The transmitter side of an aeronautical telemetry system (on board the aircraft) is generally composed of transducers that convert a physical stimulus into an electrical signal, multiplexers that combine the measurements into one signal, a transmitting equipment that modulates the data and one (or sometimes two) omnidirectional "blade" antenna(s). When the flight test range is going to be beyond $100$ km or when higher link reliability is required, a power amplifier can also be added at the transmitter side \cite{stacey2008aeronautical}. As for the receiver side (ground station), it is generally composed of a high gain parabolic dish antenna that allows tracking the aircraft, a low-noise amplifier that amplifies the signal without significantly decreasing the signal-to-noise ratio (SNR), a receiver demodulator that converts the modulated signal back to data, a recorder, and a demultiplexer.          

The design and the performance of the transmitting/ receiving equipment are mainly defined by the adopted modulation that conveys the data. The first telemetry systems were analog and used frequency modulation (FM) of analog waveforms instead of amplitude modulation (AM) \cite{osti_1143131}. This choice favored the use of power amplifiers at their saturation mode without distorting the FM constant envelope signal and provided better bandwidth occupancy/performance trade-off than AM-based systems. Then, analog FM has been progressively upgraded to digital FM by feeding the transmitter with filtered digital data instead of analog waveforms. This change only required slight modifications of the infrastructure and it did not impact the bandwidth occupancy \cite{osti_1143131}. The digital FM version is known as the non-return to zero (NRZ) pulse code modulation/frequency modulation (PCM/FM), and it can be considered as continuous phase frequency shift keying (CPFSK) modulation or more generally as continuous phase modulation (CPM) \cite{digital_phase_mod}.       

PCM/FM has been the reference modulation for decades \cite{irig106} because it has proven its robustness as well as its reliability for such high-risk use and because the required data rates were low compared to the available bandwidth before the 1990's. In the telemetry literature, this modulation is sometimes called the "legacy modulation" and is still used nowadays in several telemetry setups all over the world.  However, over the past few years, two major factors have impacted the telemetering community. The first one is the exponential expansion of wireless communication systems and especially of the commercial mobile services. As an illustration, the number of mobile-broadband subscriptions has grown more than $20\%$ annually in the last five years and has reached $4.3$ billion globally by the end of 2017 \cite{itu_facts}. This huge increase has urged telecommunication service providers to seek more bandwidth and has consequently made RF spectrum more and more costly (for instance, the United States (US) government auctioned $25$ MHz of the L-band at a record of $44.89$ billion dollars in 2015 \cite{spec_pr}). As a result, the spectrum of aeronautical telemetry has been shortened by almost $35\%$ and reallocated for the commercial applications \cite{ninja_telem}.\newline  
The second factor is the data deluge of aeronautical telemetry, which is due to the use of multiple sensors on board the aircraft to acquire more and more information during the flight. This data increase comes also as a consequence of reducing the flight testing duration for cost-effectiveness. In fact, significant resources are devoted during the flight testing phase and cost around $50$ thousand dollars in labor per hour \cite{telemetry_eco, eco_imp_tel}. Therefore, reducing the test duration without decreasing the amount of transferred data is a major demand for aerospace companies. On the one hand, this significant increase in data transfer requires the adoption of more spectrally efficient modulations than PCM/FM. On the other hand, it broadens the signal bandwidth, which makes the aeronautical telemetry channel frequency selective \cite{channel_wideband} and therefore impacts the availability of the aeronautical telemetry link. The latter issue is a real burden for aerospace companies since one delayed or cancelled test mission due to the unavailability of telemetry link could cost more than $1$ million dollars \cite{eco_imp_tel}.        

Both economical and technical challenges have stimulated the flight test telemetry community to adopt more power and bandwidth efficient modulations than PCM/FM. This has been done via the advanced range telemetry (ARTM) program \cite{artm_prog}, which resulted in $2$ spectrally efficient CPM modulations: the shaped offset quadrature phase shift keying telemetry group (SOQPSK-TG) modulation and ARTM CPM. These modulations figure in the inter-range instrumentation group (IRIG) 106 standard for aeronautical telemetry, which ensures the interoperability of telemetry systems at the range commanders council (RCC) member ranges \cite{irig106}. Both modulations are transmitter friendly since they have a constant complex envelope and require minimal infrastructure change. Moreover, they allow increasing the spectral efficiency by at least a factor of $2$ compared to PCM/FM while keeping excellent power efficiency. These attractive benefits have been achieved by carefully choosing the parameters that define a CPM. However, they come at the expense of increasing the receiver complexity, especially when considering complicated scenarios such as the presence of multipath or the use of multiple-antenna transmitters \cite{jensen}. This complexity is mainly due to two factors. The first one is the presence of an inherent memory that ensures the continuity of the phase of the modulation, and the second one is the nonlinear nature of CPM since the aforementioned modulation is not a linear function of the transmitted symbols. Therefore, unlike linear modulations (such as M-PSK, M-quadrature amplitude modulation (M-QAM), amplitude and phase-shift keying (APSK),...), it is not straightforward to take advantage of the properties of linear algebra to develop reduced complexity receivers.    
   
Even if PCM/FM is still widely used in aeronautical telemetry, we observe that more and more actors, especially in the US military ranges, migrated or started the migration to the SOQPSK solution. In such a context, the telemetry industry should develop new solutions and algorithms in order to guarantee the best possible telemetry link for these highly demanding customers. This work, supervised by Zodiac Data Systems which is one of the leading companies in aeronautical telemetry for civil and military applications, enrolls in this prospective. In this thesis, we mainly focus on SOQPSK-TG whose use is getting more and more popular in aeronautical telemetry. This modulation belongs to a particular CPM family since it transmits ternary symbols instead of binary ones. Simultaneously, it can also be interpreted as an offset modulation, i.e., it can be seen as a modulation whose quadrature component contains a timing offset compared to the in-phase component. The goal of this work is to develop reception algorithms for SOQPSK-TG under different scenarios to ensure the availability of aeronautical telemetry link. These solutions should maintain a good power efficiency without prohibitive complexity for real-time implementation.

\section*{Previous work on SOQPSK-TG}
\subsection*{SOQPSK-TG signal representations and approximations}
SOQPSK-TG is a ternary CPM whose ternary symbols are linked to the binary bits using a specific precoder given in \cite{irig106} and whose modulation index $h=1/2$. The role of the precoder is to make the phase of SOQPSK-TG behaves like the phase of an OQPSK signal when both are driven by the same bit sequence \cite{these_perrins}. As for the pulse shape that modulates the ternary symbols in the instantaneous frequency, it spans several symbol periods. The choice of these parameters increases the spectral efficiency at the expense of creating long memory and inter-symbol interference (ISI). Thus, if we take this CPM definition, the resulting reception algorithms would be highly complex. Moreover, they would be restricted to simple scenarios due to the nonlinear nature of CPM. As a consequence, it is necessary to view this modulation differently to make it more versatile.

The first pioneering work in this direction has been done by E. Perrins in \cite{these_perrins} where he showed that SOQPSK-TG could be written as a finite sum of pulse-amplitude modulated (PAM) components. The pulses of this decomposition, namely PRD, modulate pseudo-symbols. These pseudo-symbols contain memory and are nonlinear functions of the ternary symbols. This work is the extension of the Laurent decomposition \cite{pam_representation_binary}, initially developed for binary CPM. The advantage of this approach is that it approximates the signal as a sum of only two PAM waveforms. However, this does not completely "linearize" the signal because of mapping between the ternary symbols and the pseudo-symbols, which keeps the long memory effect in the approximation. 

The second work has been carried by T. Nelson in \cite{these_nelson} where he took advantage of the offset nature of SOQPSK-TG and created a direct mapping between the binary bits (i.e., the bits that are used to generate the ternary symbols) and the different waveforms that constitute the eye-diagram of the signal. The representation that makes this connection (which is not evident unlike linear modulations) is known as the cross-correlated trellis coded quadrature modulation (XTCQM) representation. After some mathematical manipulations, this representation can accurately approximate SOQPSK-TG using a bank of $8$ XTCQM waveforms, each one of them being defined by only $3$ binary bits. The advantage of this approach is that it offers an approximation of SOQPSK-TG with a short inherent memory. However, it generates data dependent waveforms, which means that we cannot separate the contribution of the bits and the pulse shape in a clear way.  

Each representation of SOQPSK-TG has its own benefits and drawbacks. Therefore, the choice between one of them depends on the desired application. 
    
\subsection*{Reduced complexity detection algorithms}
Several reduced detection algorithms have been proposed in the literature for SOQPSK-TG. The first ones are linear integrate and dump (I$\&$D) detectors. They can be used to detect the binary symbols thanks to the presence of the SOQPSK-TG precoder (to the best of our knowledge, the reason why is not analytically explained in the literature). Several I$\&$D detector implementations are given in \cite{soqpsk_opt_lin_det}, but they are sub-optimal. The best one of them suffers from a performance loss of $0.72$ dB for a probability of error of $10^{-5}$ compared to the optimal case. Then, very near-optimal performance detectors have been developed, and they rely on the presented above approximations. The first one is presented in \cite{demod_perrins} and uses the PAM approximation of PRD and operates using a dynamic trellis of $4$ states. The different transitions and the order of the path states in the this trellis depend on the parity of bit index. The second very near-optimal reduced complexity detector uses a trellis composed of $16$ states and relies on the XTCQM representation of SOQPSK-TG. The performance loss of these detectors is within $0.2$ dB compared to the optimal case.      
\subsection*{Multipath mitigation techniques}
Due to the bandwidth expansion of SOQPSK-TG signal as highlighted above, the aeronautical telemetry channel becomes frequency selective. As a consequence, the receiver may capture several delayed copies of the transmitted signal with different attenuations. The transmission is thus made over a multipath channel, which can result in significant performance loss if we use the described above detectors without introducing multipath mitigation techniques, i.e., equalizers. Channel estimation and equalization is very sparsely addressed in the literature in the SOQPSK case, contrary to other modulations. To the best of our knowledge, very few works \cite{buy_channel_taps2,CMA_soqpsk,CMA_soqpsk2} have focused on multipath mitigation techniques for SOQPSK-TG, and they mainly rely on the constant modulus algorithm (CMA) due to the constant envelope nature of this modulation. The performance of CMA can sometimes be unsatisfactory for some channel configurations. 
\subsection*{The two-antenna problem on aeronautical telemetry}
The previous architectures mentioned above are built regardless of the number of transmitting antennas. However, this parameter has an important impact on both transmitter and receiver architectures. The first aeronautical telemetry systems used one transmitting antenna fixed underneath the aircraft fuselage and one receiving antenna on a fixed ground station. However, this scheme does not guarantee an optimal telemetry link because the receiver cannot always capture the line of sight (LOS) path when the aircraft performs certain maneuvers. To overcome this situation, two antennas are then placed on the aircraft in a way that an omnidirectional transmission is guaranteed no matter how the plane is oriented. Nevertheless, if the same signal is transmitted over both antennas using the same carrier frequency, the receiver may capture both copies in a destructive way, which causes the loss of the aeronautical telemetry link. This problem is known as the two-antenna problem in aeronautical telemetry \cite{two_ant_pblm1}. A solution has been proposed in \cite{jensen}, which consists of creating transmit diversity using Alamouti space-time block coding (STBC) \cite{alamouti}. This solution significantly reduces the mutual interference of the SOQPSK-TG signals and has been recently standardized in IRIG-106 \cite{irig106}. However, it makes recovering the data bits a hard task because of the used modulation as well as the presence of an unusual channel impairment, which is the differential delay. This perceptible delay represents a non-negligible portion of the symbol peiod as explained in details in Chapter \ref{chapter:5}. It appears since both signals arrive with different delays. Several decoding architectures relying on the XTCQM representation have been proposed in \cite{these_nelson, stc_decoder} but only one of them offers satisfying performance at the expense of relatively high implementation complexity. Moreover, the best decoder only provides hard outputs and does not contain any mechanism that allows mitigating multipath in the space-time coding (STC) scenario. 

\section*{Contributions}
This dissertation addresses all the aforementioned points and proposes different solutions for each scenario. The important contributions are summarized as follows:
\begin{itemize}
\item A new PAM decomposition has been developed for SOQPSK-TG. It allows approximating the signal as a simple linear modulation and providing a straightforward mapping between the binary bits and the pseudo-symbols of this decomposition. This decomposition, namely DBD in this manuscript, is the root of all the proposed solutions of this work. 
\item A mathematical link has been developed between this proposed PAM decomposition and the XTCQM representation for SOQPSK.  
\item Several reduced complexity detectors have been proposed for SOQPSK-TG based on DBD. Each class of detectors relies on a certain approach (Kaleh approach \cite{simple_coherent}, Ungerboeck approach \cite{mlse_pam} and Forney approach \cite{forney_mlse}).
\item To combat multipath interference, a joint channel estimation and detection algorithm has been developed for SOQPSK-TG. It relies on the proposed decomposition and the per-survivor processing (PSP) principle \cite{polydoros_psp} and offers very attractive bit error rate (BER) performance for different channel configurations.
\item Two different decoding structures have been proposed for the STC scenario without changing the encoder structure. The first one offers a complexity $8$ times less than the state of the art decoder with very acceptable BER performance in the presence of the differential delay. The second one outperforms the state of the art decoder while keeping a lower complexity. Both proposed solutions can provide soft outputs (i.e., log-likelihood ratios).    
\item A new multipath channel estimator has been developed for the STC scenario and can be used along with the second proposed decoder. 
\item A new STC scheme has been developed that offers a better transmission rate for SOQPSK-TG than the IRIG standardized one. Moreover, the resulting decoding structure can process the even bits separately from the odd ones in the presence of non-zero differential delay and offers very attractive BER performance (within $1.1$ dB of the single input single output (SISO) bound). The complexity of the proposed decoder is at least $8$ times less than the state of the art decoder associated with the standardized encoding.   
\end{itemize}

\section*{Dissertation outline}
This dissertation is organized as follows.

We remind in \textbf{Chapter \ref{chapter:1}} the CPM model and we define its main parameters. We then describe the different aeronautical telemetry modulations and we highlight how the spectral efficiency is achieved for each modulation. We then focus on SOQPSK and we present the state of the art representations and approximations of this signal (i.e, the PAM decomposition and the XTCQM representation as well as their respective approximations).  

In \textbf{Chapter \ref{chapter:2}}, we first introduce a new vision of the precoder structure by decomposing it into two different stages: the first stage is a recursive encoder and the second one is a duobinary encoder. We then exploit this new vision to develop a new PAM decomposition of duobinary encoded CPM and we apply the obtained results to SOQPSK. We also establish in this chapter a link between the proposed PAM decomposition (DBD) of SOQPSK and its XTCQM representation as well its OQPSK interpretation. Furthermore, we detail the exact role of each stage that constitutes the SOQPSK precoder.

In \textbf{Chapter \ref{chapter:3}}, we build several detection architectures as a direct consequence of proposing a new PAM decomposition. We take advantage of several approaches of the literature. Some rely on the CPM definition of SOQPSK and then exploit the PAM decomposition (Kaleh approach \cite{simple_coherent} and others use the proposed "linearized" model of SOQPSK (Ungerboeck approach \cite{mlse_pam} and Forney approach \cite{forney_mlse}). A complexity study of all the proposed detectors is also carried in this chapter. 

In \textbf{Chapter \ref{chapter:4}}, we consider the presence of multipath in the SISO scenario. We first make a classification of the existing aeronautical telemetry channels of the literature. We then present the CMA solution and its different variants. We later propose a new joint channel estimation and detection algorithm for SOQPSK-TG. This solution is an adapted version of the PSP principle and relies on the proposed PAM decomposition given in Chapter \ref{chapter:2}.

In \textbf{Chapter \ref{chapter:5}}, we focus on the IRIG standardized transmit diversity scheme that allows resolving the two-antenna problem. After introducing the IRIG standardized encoder structure and the state of the art decoders, we develop two decoding structures and we present their different variants. We then consider the presence of multipath in the STC scenario in the second part of this chapter, and we detail the derivation of the new multipath estimator.     

The results given in \textbf{Chapter} \ref{chapter:5} show that the performance of proposed decoders are angle dependent due to employing the standardized scheme. For this reason, we propose in \textbf{Chapter \ref{chapter:6}} a new encoding scheme that almost eliminates this angle dependency. Moreover, the scheme ameliorates the transmission rate and allows further complexity reduction of the decoding algorithm. In the second part of this chapter, we highlight that space-time coding is not the only way to overcome the two-antenna problem. We show that sending a SOQPSK-TG signal over one antenna and its artificially delayed copy over the other can also greatly reduce their mutual interference. After determining the necessary delay that should be applied, we show that the proposed interference mitigation technique given in Chapter \ref{chapter:4} can also be used in this scenario. Thus, the same receiver architecture can be employed whether one or two antennas are mounted on the aircraft using this approach.  

%Despite the extensive literature regarding channel estimation and equalization techniques for linear modulations, it is not the case for SOQPSK-TG due to its nonlinear nature.
