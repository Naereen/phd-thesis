% Template to test TikZ figures
\documentclass[10pt,a4paper]{article}
\usepackage[hmargin=2cm,vmargin=1cm]{geometry}
\usepackage[english]{babel}
\usepackage[utf8]{inputenc}
\usepackage{tikz}
\usetikzlibrary{arrows,decorations.pathmorphing,backgrounds,fit,positioning,shapes.symbols,chains}
\usepackage{verbatim}
\usepackage[active,tightpage,floats]{preview}
\setlength\PreviewBorder{25pt}%
%%% BEGIN DOCUMENT
\begin{document}

% \begin{figure}[h!]
% \begin{tikzpicture}[>=latex',line join=bevel,scale=3]
%     %
%     \node[align=center] (introduction) at (0,2.5) [rectangle,draw] {Chapter~\ref{chapter:1}\\Introduction};
%     \node[align=center] (chapter2) at (0,2) [rectangle,draw] {Chapter~\ref{chapter:2}\\Multi-Armed Bandit models};
%     \node[align=center] (chapter3) at (3,2) [rectangle,draw] {Chapter~\ref{chapter:3}\\SMPyBandits: simulation library for MAB};
%     \node[align=center] (chapter4) at (-2,1) [rectangle,draw] {Chapter~\ref{chapter:4}\\Two MAB models for IoT networks};
%     \node[align=center] (chapter5) at (0,1) [rectangle,draw] {Chapter~\ref{chapter:5}\\Multi-players MAB};
%     \node[align=center] (chapter6) at (2,1) [rectangle,draw] {Chapter~\ref{chapter:6}\\Non-stationary MAB models};
%     \node[align=center] (conclusion) at (0,0) [rectangle,draw] {Chapter~\ref{chapter:conclusion}\\General conclusion};
%     %
%     \draw [color=black,thick,->] (introduction) to (chapter2);
%     \draw [color=black,thick,->] (chapter2) to (chapter3);
%     \draw [color=black,thick,->] (chapter2) to (chapter4);
%     \draw [color=black,thick,->] (chapter2) to (chapter5);
%     \draw [color=black,thick,->] (chapter2) to (chapter6);
%     \draw [color=black,thick,->] (chapter4) to (conclusion);
%     \draw [color=black,thick,->] (chapter5) to (conclusion);
%     \draw [color=black,thick,->] (chapter6) to (conclusion);
%     %
% \end{tikzpicture}
% \caption{Organization of the thesis.}
% \label{fig:1:organization}
% \end{figure}

\tikzstyle{block} = [align=center, draw, fill=gray!25, rectangle, minimum height=3em, minimum width=6em]

\begin{figure}[h!]
    \centering
\begin{tikzpicture}[auto,node distance=5cm,>=latex,scale=2]
    %
    % We start by placing the blocks
    \node [block] (player) at (0,0) {Player};
    % We draw an edge between the player and system block to
    \node [block] (environment) at (2,0) {Environment};
    % Once the nodes are placed, connecting them is easy.
    \draw [->] (player) to[bend left=90] node[pos=0.5] {Action} (environment);
    \draw [->] (environment) to[bend left=90] node[pos=0.5] {Reward} (player);
    %
\end{tikzpicture}
\caption{Reinforcement learning cycle.}
\label{fig:1:organization}
\end{figure}

\end{document}