% ************************** Thesis Abstract *****************************
% Use `abstract' as an option in the document class to print only the titlepage and the abstract.
% \vspace*{-30pt}
%
\begin{abstract}
% \vspace*{-30pt}

% ----------------------------------------------------------------------------
\begin{center}
	\textbf{{\LARGE Résumé}}
\end{center}

%
% \TODOL{Raccourcir juste un peu ?}
% ----------------------------------------------------------------------------
% \begin{small} % XXX
% Chapitre 1
Dans cette thèse de doctorat, nous étudions les réseaux sans fil et les appareils reconfigurables qui peuvent accéder à des réseaux de radio intelligente, dans des bandes non licenciées et sans supervision centrale.
Nous considérons des réseaux de l'Internet des Objets (IoT), avec l'objectif d'augmenter la durée de vie de la batterie des appareils, en les équipant d'algorithmes d'apprentissage machine peu coûteux mais efficaces, qui leur permettent d'améliorer automatiquement l'efficacité de leurs communications sans fil.
%  (Chapitre~\ref{chapter:1}).
% Chapitre 4
Nous proposons différents modèles de réseaux IoT, et nous montrons empiriquement, par des simulations numériques et une validation expérimentale réaliste, le gain que peuvent apporter nos méthodes, qui utilisent l'apprentissage par renforcement.
%  (Chapitre~\ref{chapitre:4}).
% Chapitre 2
Les différents problèmes d'accès au réseau sont modélisés avec des Bandits Multi-Armés (MAB), mais leur analyse est difficile à réaliser,
% , Chapitre~\ref{chapitre:2}
car il est difficile de prouver la convergence de nombreux appareils jouant à un jeu collaboratif sans communication ni aucune coordination, lorsque les appareils suivent tous un modèle d'activation aléatoire.
% En effet, même le nombre de périphériques actifs à chaque instant est hautement imprévisible, rendant l'environnement non pas stationnaire et évoluant rapidement.
% Chapitres 5 et 6
Le reste de ce manuscrit étudie donc deux modèles restreints, d'abord des bandits multi-joueurs dans des problèmes stationnaires, puis des bandits mono-joueur non stationnaires.
%  (Chapitre~\ref{chapitre:5})
%  (Chapitre~\ref{chapitre:6})
% Chapitre 3
Nous détaillons également une autre contribution, la bibliothèque Python open-source SMPyBandits pour des simulations numériques de problèmes MAB, qui couvre les modèles étudiés et d'autres.
% (Chapitre~\ref{chapter:3})
% \end{small} % XXX

% ----------------------------------------------------------------------------
\hr{}

% \newpage
% ----------------------------------------------------------------------------
\begin{center}
	\textbf{{\LARGE Abstract}}
\end{center}
% \vspace*{1cm}
% \vspace*{-5pt}
%
% \begin{small} % XXX
% Chapter 1
In this PhD thesis, we study wireless networks and reconfigurable end-devices that can access large-scale Cognitive Radio networks, in unlicensed bands and without central control.
We focus on Internet of Things networks (IoT), with the objective of extending the devices' battery life, by equipping them with low-cost but efficient machine learning algorithms, to let them automatically improve the efficiency of their wireless communications.
%  (Chapter~\ref{chapter:1})
% Chapter 4
We propose different models of IoT networks, and we show empirically on both numerical simulations and real-world validation the possible gain of our methods, that use Reinforcement Learning.
%  (Chapter~\ref{chapter:4})
% Chapter 2
The different network access problems are modeled as Multi-Armed Bandits (MAB), but we found that analyzing the realistic models was intractable,
% ,  Chapter~\ref{chapter:2}
because proving the convergence of many end-devices playing a collaborative game without communication no coordination is hard, when end-devices all follow random active pattern.
% Indeed, even the number of active devices at any time is highly unpredictable, making the environment not stationary and quickly evolving.
% Chapter 5 and 6
The rest of this manuscript thus studies two restricted models, first multi-players bandits in stationary problems, then non-stationary single-player bandits.
%  (Chapter~\ref{chapter:5})
%  (Chapter~\ref{chapter:6})
% Chapter 3
We also detail another contribution, SMPyBandits, an open-source Python library for numerical MAB simulations, covering the studied models and more.
%  (Chapter~\ref{chapter:3})
% \end{small} % XXX


\end{abstract}
