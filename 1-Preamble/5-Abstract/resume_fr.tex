\begin{resume_fr}

% Résumé de la thèse en français.

% ----------------------------------------------------------------------------

Ce manuscrit conclut ma thèse de doctorat, qui a débuté en octobre $2016$ et s'est terminée en novembre $2019$.
Mes recherches se sont déroulées au laboratoire IETR à Rennes en France, dans l'équipe SCEE hébergée sur le campus de Rennes de l'école d'ingénieurs CentraleSupélec.
J'étais supervisé par le professeur Christophe Moy, à Rennes,
et j'étais également co-encadré par la docteure Émilie Kaufmann, à qui j'ai souvent rendu visite à Inria Lille Nord Europe.


% % Chapitre 1
% Dans cette thèse de doctorat, nous étudions les réseaux sans fil et les objets reconfigurables qui peuvent accéder à des réseaux de radio sans fil de manière intelligente, dans des bandes non licenciées et sans supervision centrale.
% Plus spécifiquement, nous considérons des réseaux de l'Internet des Objets (IdO), avec l'objectif d'augmenter la durée de vie de la batterie des objets, en les équipant d'algorithmes d'apprentissage machine peu coûteux mais efficaces, qui leur permettent d'améliorer automatiquement l'efficacité de leurs communications sans fil (Chapitre~\ref{chapter:1}).
% % Chapitre 4
% Nous proposons différents modèles de réseaux de l'IdO, et nous montrons empiriquement, par des simulations numériques et une validation expérimentale réaliste, le gain que peuvent apporter nos méthodes, qui utilisent l'apprentissage par renforcement (Chapitre~\ref{chapter:4}).
% % Chapitre 2
% Les différents problèmes d'accès au réseau sont modélisés avec des Bandits Multi-Bras (BMB, ou \emph{Multi-Armed Bandits}, MAB, en Chapitre~\ref{chapter:2}), mais leur analyse est complexe,
% % , Chapitre~\ref{chapter:2}
% notamment si l'on veut prouver la convergence de nombreux objets jouant à un jeu collaboratif sans aucune coordination, lorsque les objets suivent tous un profil d'activation aléatoire.
% % En effet, même le nombre de périphériques actifs à chaque instant est hautement imprévisible, rendant l'environnement non pas stationnaire et évoluant rapidement.
% % Chapitres 5 et 6
% Le reste de ce manuscrit étudie donc deux modèles restreints, afin de pouvoir en faire une analyse théorique exhaustive.
% Nous considérons d'abord des bandits multi-joueurs dans des problèmes stationnaires (Chapitre~\ref{chapter:5}), puis des bandits mono-joueurs non stationnaires (Chapitre~\ref{chapter:6}).
% % Chapitre 3
% Nous détaillons également une autre contribution, la bibliothèque Python open-source SMPyBandits pour des simulations numériques de problèmes BMB, qui couvre tous les modèles étudiés et d'autres (Chapitre~\ref{chapter:3}).


% ----------------------------------------------------------------------------
\section*{Contexte de cette thèse}

% FIXME 0.5 à 2 pages de contexte, historique, technique, humain, etc.

% Problème principal
%
Les problèmes sous-jacents qui motivent cette thèse sont les questions du réchauffement climatique et de l'augmentation de la population mondiale.
% https://en.wikipedia.org/wiki/History_of_radio
Au cours des $150$ dernières années, l'humanité a développé de nombreuses technologies de communication différentes, et depuis la fin des années $1890$, les télécommunications sans fil entre des appareils fabriqués par l'homme ont été rendues possibles, et de plus en plus fréquentes dans nos vies.
Avec l'avènement des réseaux de l'Internet des Objets (IdO), des milliards d'objets à basse consommation devraient être déployés dans le monde entier, permettant un large éventail de nouvelles applications.
Il existe aujourd'hui un consensus mondial sur le fait qu'avec la tendance actuelle à l'accroissement démographique et la crise énergétique, toute nouvelle technologie déployée doit être à la fois bon marché et efficace sur le plan énergétique,
ainsi qu'adaptée pour desservir un grand nombre de personnes et d'appareils.
%
Ces technologies des nouveaux réseaux sans fil de l'IdO devraient pouvoir s'adapter automatiquement à différents environnements et différents contextes d'application, et être aussi efficaces que possible.
%
C'est pourquoi, en plus de l'effort habituel de recherche et développement pour concevoir de nombreux systèmes d'accès radio efficaces, couvrant tous les cas possibles,
le moment est venu de le combiner avec un apprentissage machine peu coûteux, dans le but d'atteindre le niveau de gain de performance nécessaire pour que les promesses de l'IdO deviennent réalité.


C'est ainsi que nous avons décidé de nous intéresser dans cette thèse à
l'amélioration de la durée de vie des batteries des objets de l'IdO et la réduction du coût énergétique des réseaux de l'IdO.
Nous proposons d'atteindre ces deux objectifs conjointement, en intégrant une prise de décision décentralisée à faible coût, directement dans les futurs objets de l'IdO.

Ma thèse de doctorat porte donc sur les applications possibles de l'intégration d'un certain type d'algorithmes d'apprentissage machine (des algorithmes de type bandits multi-bras), afin de permettre aux objets de l'IdO d'optimiser leurs communications sans fil et d'apprendre à s'organiser automatiquement et sans contrôle central ni coordination.


\paragraph{Des anciens téléviseurs aux standards de l'IdO.}
%
Historiquement, trois grandes familles de systèmes de communication sans fil ont été déployées dans les grands réseaux commerciaux : d'abord, la radiodiffusion centralisée (\eg, radio ou télédiffusion), puis les systèmes bidirectionnels centralisés (\eg, 4G ou Wi-Fi), et aujourd'hui la collecte décentralisée de données pour les réseaux de l'Internet des objets (\eg, réseaux capteurs).
%
Ce troisième type de systèmes peut être désigné comme décentralisé :
même si une station de base centrale est toujours en charge de nombreux objets (ou appareils),
ce sont les objets qui déclenchent l'envoi de paquets radio, et les seules données descendantes qu'ils peuvent recevoir sont de courts accusés de réception (ou acquittements, en anglais \emph{acknowledgements}, noté \Ack), envoyés par la station de base pour indiquer le succès ou l'échec de chaque paquet envoyé.
Cette famille de systèmes sans fil est appelée Internet des Objets (IdO, ou \emph{Internet of Things}, IoT),
et un exemple typique d'application de tels réseaux de l'IdO est celui des réseaux sans fil de capteurs.

Pour le développement futur des \guillemotleft{} réseaux intelligents \guillemotright{} (\emph{smart grid}), des \guillemotleft{} villes intelligentes \guillemotright{} (\emph{smart cities}), des \guillemotleft{} maisons intelligentes \guillemotright{} (\emph{smart homes}) ou de \guillemotleft{} l'agriculture intelligente \guillemotright{} (\emph{smart agriculture}), des réseaux de capteurs devraient être largement déployés.
Deux exemples d'applications futures qui sont déjà en cours de déploiement, en France ou à l'étranger, sont les \guillemotleft{} bâtiments connectés \guillemotright{} et \guillemotleft{} l'agriculture connectée \guillemotright{}.
Pour les bâtiments, l'objectif principal est de réduire le coût de chauffage des bâtiments vides en utilisant des réseaux de capteurs afin d'obtenir des données précises et régulières sur la température dans toutes les pièces et tous les étages, pour permettre au contrôle centralisé du chauffage de minimiser son coût et sa consommation énergétique.
Pour l'agriculture, un exemple peut être d'équiper chaque tête de bétail (dans les grandes fermes) de capteurs qui émettent régulièrement des informations biologiques, telles que la température corporelle ou le niveau de stress, afin d'optimiser l'heure de traite des vaches, de surveiller la santé des animaux, etc.


Ce troisième type de systèmes sans fil est caractérisé par sa nature décentralisée,
puisque les communications sont initialisées et régulées par les objets, et non par un système de contrôle centralisé.
%
En effet, un contrôle central nécessite des paquets de signalisation très réguliers, qui ont été identifiés comme trop lourds pour ce type de systèmes.
Dans les réseaux de l'IdO actuels et futurs, de nombreux objets hétérogènes utilisent la même station de base (aussi appelée point d'accès, ou \emph{base station}, \emph{access point} ou \emph{gateway}) pour des applications différentes.
Un problème commun est la forte contrainte en terme de consommation énergétique de ces objets, car la plupart d'entre eux seront déployés sans alimentation directe et fonctionneront sur une batterie minuscule, dont la durée de vie devrait être maximisée.
Ainsi, la plupart des entreprises promettant des solutions d'IdO vendent aujourd'hui des objets ayant une durée de vie supérieure à $10$ ans, comme SIGFOX \cite{Centenaro16}.
Une autre contrainte commune aux objets de l'IdO est leur faible besoin en communication, car la plupart des applications n'auront besoin d'envoyer qu'un ou quelques messages chaque jour, en nette opposition avec le débit de données élevé recherché pour les systèmes centralisés (tels que 4G/5G et Wi-Fi).
%
De nombreuses normes différentes pour les réseaux de l'IdO ont été proposées ces dernières années,
et elles consistent en une spécification à la fois pour la couche \emph{PHY}sique
et la couche \emph{M}edium \emph{A}ccess \emph{C}ontrol (\emph{MAC}).
Pour citer quelques exemples de normes pouvant être utilisées pour des réseaux de type IdO, ZigBee, Z-Wave ou Bluetooth visent des communications à courte portée (jusqu'à \SI{2}{\meter}), tandis que LoRaWAN, SIGFOX, Ingenu ou Weightless sont destinés aux communications à plus longue portée (jusqu'à \SI{50}{\kilo\meter}).
Nous référons à l'article \cite{Centenaro16} pour plus de détails, et les références dans \cite{Azari18} ou notre dernier travail \cite{MoyBesson2019Annales}.


\paragraph{Épuisement du spectre radio.}
%
Un problème majeur des technologies sans fil actuelles est la question de l'épuisement du spectre radio :
dans la plupart des bandes de fréquences, la totalité du spectre de radio fréquences (RF) est désormais attribuée et les bandes libres n'existent plus, ce qui limite la possibilité d'ajouter de nouveaux usages.
Comme le montre la Figure~\ref{fig:1:United_States_Frequency_Allocations_Chart_2016_The_Radio_Spectrum} ci-dessous,
presque tout le spectre
est affecté à divers usages, qui vont de la radio-navigation maritime (historiquement le premier usage des radio-communications, \eg, dans le Titanic), à la recherche spatiale, les communications inter satellitaire, la téléphonie mobile et de nombreuses autres applications.
%
Les organismes de réglementation dans le monde, comme l'
\href{https://www.itu.int/en/Pages/default.aspx}{Union Internationale des Télécommunications} (\emph{ITU}, voir \href{https://www.itu.int/}{\texttt{www.ITU.int}}),
la \href{https://www.fcc.gov/}{Commission Fédérale des Communications} aux États-Unis d'Amérique du Nord (\emph{FCC}, voir \href{https://www.fcc.gov/}{\texttt{FCC.gov}})
ou la \href{https://cept.org}{Conférence Européenne des Administrations des Postes et des Télécommunications} en Europe (\emph{CEPT}, voir \href{https://www.CEPT.org/}{\texttt{CEPT.org}}),
ainsi que différentes campagnes de mesure indépendantes, ont montré que la plupart des fréquences du spectre des fréquences radio-électriques sont utilisées de manière inefficace.
Cela signifie qu'une bande peut être attribuée à un certain usage unique, mais qu'elle peut être libre de tout utilisateur à certains moments et/ou endroits.
Nous nous référons à \cite{patil2011survey} pour une étude sur l'utilisation mondiale du spectre, et à \cite{valenta2010survey} pour la situation en Europe.


Les bandes des réseaux cellulaires (2G/3G/4G/5G) sont surchargées dans la plupart des régions du monde, mais d'autres bandes de fréquences (comme les fréquences militaires ou de radio amateures) sont moins utilisées.
Des études indépendantes réalisées dans certains pays ont confirmé cette observation et conclu que l'utilisation du spectre peut fortement dépendre à la fois du moment et du lieu, comme le montre \cite{Lopez2009spectral} par exemple.
En outre, l'attribution fixe du spectre empêche
l'introduction de nouveaux services, en particulier pour les objets à bas prix ou pour les marchés de niche.
%
C'est ainsi qu'au cours des quinze dernières années, grâce à un lobbying actif de la communauté de la radio intelligente,
les organismes de réglementation dans le monde se sont demandé s'il fallait permettre un nouveau paradigme de communication sans fil :
permettre aux utilisateurs non titulaires d'une licence d'utiliser des bandes sous licence s'ils ne causent pas d'interférence aux utilisateurs payants, titulaires d'une licence.
Ces initiatives sont examinées par la \textbf{Radio Intelligente} (RI, ou \emph{Cognitive Radio}, CR),
que nous détaillons ci-dessous, et en particulier pour l'\textbf{Accès Dynamique au Spectre} (ADS, ou \emph{Dynamic Spectrum Access}, DSA),
pour lesquelles nous renvoyons aux articles \cite{akyildiz2006next,garhwal2012survey} pour plus de détails.



\section*{Radio Intelligente et Bandits Multi-Bras}

Dans cette thèse, nous étudions les réseaux Internet des Objets à longue portée, caractérisés principalement par trois contraintes essentielles :
faible consommation d'énergie (ou longue durée de vie des batteries),
longue portée,
et un faible cycle d'utilisation (\ie, faible à très faible nombre de messages par jour).
%
Plus précisément, nous étudions les interconnexions possibles entre \textbf{Radio Intelligente} et \textbf{l'apprentissage statistique} appliquées aux réseaux de l'IdO.
Définissons et détaillons les deux concepts séparément.


\paragraph{Des TIC à la Radio Intelligente (RI).}
%
Nous pouvons affiner le premier champ d'étude de cette thèse, étape par étape :
des technologies de l'information et communication (TIC), aux télécommunications, aux communications sans fil, puis à la Radio Intelligente,
et enfin à la Radio Logicielle (RL, ou \emph{Software Defined Radio}, SDR).
%
La transition de l'approche historique de la radio matérielle aux architectures RL est un processus graduel, qui a commencé au début des années $1990$ et s'est accéléré dans les années $2000$.
% https://en.wikipedia.org/wiki/Software-defined_radio
Une RL est un système de radiocommunication où les composants qui ont été traditionnellement implémentés par du matériel dédiés (\eg, mélangeurs, filtres, modulateurs/démodulateurs, détecteurs, etc) sont de plus en plus implémentés au moyen de logiciels sur un processeur.
%
Même si le paradigme de la RI a été initié par des recherches de l'armée des États-Unis d'Amérique du Nord dans les années $1980$, l'industrie civile a commencé à s'intéresser à la RI au cours des vingt dernières années, et la RI a également suscité beaucoup d'intérêt de la part du milieu universitaire.
%
Comme la RI n'est pas une technologie standard, elle n'a pas de définition unique, commençons donc par citer la définition de deux chercheurs dont les travaux ont été à l'origine du développement de la RI, dont le premier vient de paraître il y a vingt ans.
%
\begin{itemize}\tightlist
    \item
    J. Mittola, en $1999$, a proposé que
    \guillemotleft{} \emph{une radio vraiment intelligente qui serait autonome, sensible aux RF et à l'utilisateur, et qui inclurait la technologie logicielle et les capacités d'apprentissage machine ainsi qu'une grande connaissance de l'environnement radio haute-fidélité} \guillemotright{} \cite{Mitola99}.

    \item
    Puis S. Haykin en $2005$ a aussi dit que
    \guillemotleft{} \emph{une radio intelligente (RI) est un système de communication sans fil intelligent qui est capable de connaître son environnement, d'apprendre et d'adapter ses paramètres de fonctionnement (\eg, puissance d'émission et fréquence porteuse) à la volée, dans le but de fournir une communication fiable à tout moment, en tout lieu et efficace sur le plan spectral} \guillemotright{} \cite{Haykin05}.

    \item
    L'\href{https://en.wikipedia.org/wiki/Cognitive_radio}{encyclopédie Wikipédia} dit ceci
    \guillemotleft{} \emph{une RI est une radio qui peut être programmée et configurée dynamiquement pour utiliser les meilleurs canaux sans fil à proximité afin d'éviter les interférences et la congestion des utilisateurs. Une telle radio détecte automatiquement les canaux disponibles dans le spectre radio, puis modifie en conséquence ses paramètres de transmission ou de réception pour permettre plus de communications sans fil simultanées dans une bande de spectre donnée à un endroit donné} \guillemotright{}.
    (\href{https://en.wikipedia.org/wiki/Cognitive_radio}{\texttt{en.Wikipedia.org/wiki/Cognitive\_radio}})
\end{itemize}

L'une des façons possibles d'envisager la flexibilité du spectre est la suivante.
%
Dans les bandes sous licences, il y a des utilisateurs primaires (UP) qui paient un abonnement pour accéder au réseau, par exemple n'importe qui doit payer pour avoir un numéro de téléphone mobile et utiliser le réseau.
Comme nous l'avons vu dans la définition ci-dessus de la philosophie de la CR, nous pouvons imaginer que si le réseau n'est utilisé par aucun UP à un certain endroit et à une certaine heure, les utilisateurs non licenciés, appelés utilisateurs secondaires (US), pourraient l'utiliser aussi, éventuellement en payant un abonnement auprès de l'opérateur du réseau.
%
La réglementation stipule que les UP ont une priorité stricte,
mais même si les bandes RF sont attribuées, les mesures dans le monde réel montrent souvent que certaines bandes ne sont pas utilisées de manière intensive, et donc si un US était équipé d'une capacité de détection spectrale efficace, il pourrait analyser son environnement, et utiliser une bande sous licence si et seulement si elle est exempte de tout UP.
Ceci définit le concept d'\textbf{Accès Opportuniste au Spectre} (AOS, ou \emph{Opportunistic Spectrum Access}, OSA), pour lequel nous nous référons à l'article \cite{Zhao07} pour plus de détails.


\paragraph{Des statistiques ou de l'apprentissage machine aux bandits multi-bras.}
%
Nous sommes intéressés par l'apprentissage séquentiel et en particulier par les Bandits Multi-Bras (BMB),
qui sont apparus comme des problèmes intéressants au sein de la communauté des statistiques et de l'apprentissage séquentiel, comme le montrent les travaux pionniers de \cite{Thompson33,Robbins52,LaiRobbins85}.
Les BMB sont également inclus dans le concept plus général d'apprentissage par renforcement (ApR, ou \emph{Reinforcement Learning}, RL), lui-même un des domaines de l'apprentissage machine (AM, ou \emph{Machine Learning}, ML).
% ML > RL > MAB
%
Le livre de référence sur l'ApR est \cite{SuttonBarto2018}, citons donc la définition qu'en donnent R. Sutton et A. Barto :
\guillemotleft{} \emph{l'apprentissage par renforcement, c'est apprendre quoi faire -- comment faire le lien entre les situations et les actions -- pour maximiser un signal de récompense numérique. L'apprenant n'est pas informé des mesures à prendre mais doit plutôt découvrir quelles actions sont les plus gratifiantes en les essayant} \guillemotright.
% apprentissage statistique > séquentiel > par renforcement > information partielle (bandits)


Nous illustrons ci-dessous l'idée d'un cycle d'apprentissage, alternant entre action et réaction,
en Figure~\ref{fig:1:ReinforcementLearningCycle_enFr}.
Un-e joueur-se (ou un-e apprenant-e) interagit avec son environnement en prenant une action $A(t)$ (\eg, un choix dans un ensemble fini, $A(t)\in\{1,\dots,K\}$, ou un vecteur $A(t)\in\R^d$), et ensuite en observant une récompense $r(t)$, qui est une certaine mesure du succès de cette action, fournie par l'environnement (\eg, $r(t)\in\{0,1\}$ pour échec/succès binaire, ou $r(t)\in\R$).
Le but de la joueuse est de maximiser ses récompenses, par des essais et des erreurs (\ie, actions et récompenses).
De nombreux problèmes du monde réel peuvent être présentés comme des problèmes d'apprentissage par renforcement, comme l'illustrent l'article \cite{bouneffouf2019survey} et la Section~\ref{sec:2:applicationsofStochasticMAB}, par exemple apprendre à marcher, à conduire, à jouer à un jeu vidéo, découvrir quel traitement est efficace pour guérir une certaine maladie, etc.



\tikzstyle{block} = [align=center, draw, fill=gray!25, rectangle, minimum height=3em, minimum width=6em]
\begin{figure}[h!]
    \centering
    \resizebox{0.30\textwidth}{!}{
        \begin{tikzpicture}[auto,node distance=5cm,>=latex,scale=1.5]
            %
            % We start by placing the blocks
            \node [block] (player) at (0,0) {Joueuse};
            % We draw an edge between the player and system block to
            \node [block] (environment) at (2,0) {Environnement};
            % Once the nodes are placed, connecting them is easy.
            \draw [->] (player) to[bend left=90] node[pos=0.5] {Action} (environment);
            \draw [->] (environment) to[bend left=90] node[pos=0.5] {Récompense} (player);
            %
        \end{tikzpicture}
    }
\caption{Cycle de l'apprentissage par renforcement : un-e joueur-se interagit avec son environnement par des actions, et observe une récompense, de façon itérative.}
\label{fig:1:ReinforcementLearningCycle_enFr}
\end{figure}


% Il faut expliquer ce que c'est que les bandits, rapidement, et leur application à l'OSA
Comme cette thèse se concentre sur les modèles d'apprentissage par renforcement,
il est important de souligner que dans de tels modèles de prise de décision, l'apprenant n'a pas accès à l'ensemble des réactions possibles de l'environnement après avoir choisi son action.
%
En d'autres termes, la joueuse ne voit que la récompense donnée par son action à chaque tour, et non la récompense qui aurait été donnée si elle avait choisi une des autres actions.
Ce genre de rétroaction limitée s'appelle l'information de type bandit (\emph{bandit feedback}), et nous discutons de l'histoire et du concept des \emph{bandits} (BMB) en détails dans le prochain Chapitre~\ref{chapter:2}.
Les essais cliniques et l'identification du meilleur traitement ont été historiquement les premières applications des BMB depuis les années $1930$, avec les premiers travaux de W. Thompson \cite{Thompson33},
%
car les BMB sont un exemple simple mais puissant du dilemme bien connu entre \emph{exploration et exploitation}.
Lorsqu'il est confronté à un ensemble de $K$ actions dont les effets sur l'environnement sont inconnus, l'apprenant doit trouver un équilibre entre
explorer les actions inconnues, afin de recueillir plus d'informations à leur sujet,
et en exploitant la meilleure action, selon ses connaissances actuelles.
%
Les problèmes de bandits ont été étudiés à la fois dans la communauté de l'apprentissage machine et de la statistique, depuis les années $1950$ avec des pionniers comme H. Robbins \cite{Robbins52}, et c'est un domaine de recherche actif depuis les années $1990$ \cite{Anantharam87a,Anantharam87b,auer1995gambling,Agrawal95}.
La recherche sur les bandits a produit une vaste littérature durant les années $2000$ \cite{Auer02,Auer02NonStochastic,Audibert2009minimax} et continue d'être un sujet actif, comme l'illustrent les livres \cite{Bubeck12,LattimoreBanditAlgorithmsBook,Slivkins2019} et les nombreuses applications des bandits les dernières années \cite{bouneffouf2019survey}.


\paragraph{Accès Opportuniste au Spectre (AOS).}
%
Les deux communautés de l'AOS et des bandits ont commencé à interagir, et les travaux qui en ont résulté ont suscité un grand intérêt de la part des deux communautés, depuis la fin des années $2000$ et le début des années $2010$, avec des travaux pionniers comme \cite{Liu08,Zhao10,Jouini09,Jouini10}.
L'accent est mis sur un US accédant à un spectre sous licence, occupé par un UP, qui a une priorité stricte sur l'US, qui doit suivre un schéma d'accès de type \guillemotleft{} écoute avant de transmettre \guillemotright{}.
%
Les hypothèses sont les suivantes :
l'US est équipé d'une capacité de détection,
et considère un ensemble fixe et fini de $K$ canaux orthogonaux, \ie, différentes bandes de fréquences dans un spectre sous licence.
Par exemple, il peut s'agir d'un ensemble de $K=3$ canaux Wi-Fi à différentes fréquences, émis par la même station Wi-Fi.
Une autre hypothèse est que UP et US sont synchronisés dans le temps, en subdivisant le temps en pas de temps discrets.
%
Ainsi, si l'US passe un peu de temps au début de chaque plage horaire pour essayer de détecter un UP, il peut rechercher la présence ou l'absence d'un UP, avant de transmettre pendant le reste de la plage horaire, sans entrer en collision avec l'UP.
Si l'US était capable d'analyser tous les $K$ canaux, il pourrait simplement émettre dans l'un des canaux libres s'il y en avait, ou ne pas émettre si tous les canaux sont utilisés à un pas de temps donné.
%
Cependant, on sait que la \emph{détection spectrale} est coûteuse en énergie, en particulier pour la détection à large bande, comme le montrent les articles \cite{yucek2009survey,subhedar2011spectrum} (\emph{spectrum sensing}). C'est pourquoi la plupart des travaux sur l'AOS limitent la capacité de détection de l'US à un seul canal à la fois.
Cette hypothèse, ainsi que l'hypothèse selon laquelle les UP ne peuvent pas être perturbés (sans laquelle aucune réglementation ne sera jamais acceptée pour l'AOS), impose à l'US de transmettre dans le canal qu'il a analysé, si et seulement s'il a été détectée sans aucun UP, et ce à chaque pas de temps.
% coût non négligeable de la reconfiguration radio

En se concentrant sur un US s'insérant dans un réseau permettant l'AOS, il doit décider séquentiellement d'un canal à analyser (dans l'ensemble des canaux, $[K]=\{1,\dots,K\}$), puis il écoute ce canal et effectue une détection de PU, et enfin il transmet dans ce canal s'il a été détecté libre.
Le but de l'US est de minimiser sa consommation d'énergie (nous rappelons que nous nous concentrons sur la radio intelligente \guillemotleft{} écologique \guillemotright{}, \emph{green radio}) et de maximiser son débit de données sur la liaison montante, ou de façon équivalente, de maximiser son nombre de transmissions réussies.
%
Si les différents canaux ne sont pas uniformément occupés par les UP, et si l'on suppose une certaine stationnarité sur le trafic des UP, alors l'objectif de l'US se résume à explorer les différents canaux et à exploiter les meilleurs.
Le problème de l'AOS est donc un problème d'exploration/exploitation, avec un ensemble fini d'actions (les canaux, également appelés \emph{bras}),
dans un cycle séquentiel d'action-réponse (les pas de temps sont $t\in\N^*$),
sous l'information de rétroaction partielle (\ie, l'US reçoit l'information concernant un seul canal parmi $K$).
Ces trois hypothèses sont celles qui limitent le cadre général de l'apprentissage séquentiel au cas spécifique des bandit multi-bras (voir Figure~\ref{fig:1:ReinforcementLearningCycle_enFr}).


\paragraph{BMB pour l'AOS, et un bref historique de ces recherches par l'équipe SCEE.}
%
Les travaux antérieurs de notre équipe SCEE ont montré que les BMB peuvent être utilisés pour modéliser le problème de prise de décision de l'AOS :
les bandes de fréquences orthogonales (ou canaux) sont modélisées par des bras $k\in\{1,\dots,K\}$,
et la récompense obtenue par l'objet après avoir analysé le canal $k$ au temps $t$ est modélisée par une récompense $r(t) \in \{0,1\}$.
En effet, $r(t) = 1$ indique qu'aucun utilisateur primaire n'a été détecté (et donc qu'un message d'US peut être envoyé), alors que $r(t)=0$ indique que le canal $k$ est occupé durant l'intervalle de temps $t$, et qu'aucun message d'US ne doit être envoyé, jusqu'à la fin de cet intervalle.
%
Ce modèle a d'abord été étudié par W. Jouini lors de sa thèse de doctorat avec C. Moy \cite{Jouini12PhD}, il y a dix ans, d'abord dans l'article \cite{Jouini09} puis avec \cite{Jouini10,Jouini12}.

Leurs travaux ont été parmi les premiers à proposer l'utilisation de l'apprentissage par renforcement pour la radio intelligente et le problème AOS, notamment le modèle BMB et l'algorithme $\UCB_1$,
en parallèle des premiers travaux de Q. Zhao et de son équipe, par exemple \cite{Liu08,Zhao10}.
%
Peu de temps après, en $2014$, C. Moy et son étudiant C. Robert \cite{RobertSDR2014,MoyWSR2014,MoyWSR2014} ont développé des preuves de concept utilisant du matériel radio réaliste et la radio logicielle, avec des cartes USRP et le logiciel MATLAB/Simulink.
Dans une seconde thèse, N. Modi a étudié de $2014$ à $2017$ l'impact sur la durée de vie des batteries d'un objet sans fil de l'utilisation des algorithmes BMB pour optimiser la sélection des canaux, \cite{Modi17PhD}.
% D'une part, l'utilisation d'un algorithme BMB tel que les algorithmes de type UCB s'est avérée utile et pourrait apporter des améliorations significatives en termes de taux de transmission, augmentant directement la durée de vie de la batterie du objet.
% D'autre part, les algorithmes BMB classiques ont tendance à changer souvent de bras, surtout au début de leur processus d'apprentissage, ce qui induit beaucoup de reconfigurations matérielles dynamiques pour l'objet sans fil, car la sélection d'un canal différent nécessite un changement dans le matériel radio utilisé par l'objet.
% Chaque reconfiguration matérielle coûte de l'énergie à l'objet, et les algorithmes à commutation rapide entraînent donc une réduction de l'autonomie de la batterie.
Le compromis entre des reconfigurations fréquentes, qui coûtent en énergie, mais qui permettent d'apprendre rapidement est étudié de façon empirique dans \cite{modiDemo2016}.


En $2017$, C. Moy a continué à travailler dans cette direction, avec un étudiant post-doctoral, S. Darak, qui a mené à des publications telles que
\cite{darak2016bayesian,Darak16}.
%
Par exemple, des preuves de concepts comme \cite{kumar2016two} ont prouvé la capacité de telles approches sur des signaux radio réels pour l'AOS.
%
Certaines analyses de réelles mesures radio effectuées pour les canaux ionosphériques HF ont également prouvé que les solutions basées sur l'apprentissage BMB sont appropriées et résolvent efficacement ce type de problèmes de prise de décision sur les signaux sans fil du monde réel \cite{Melian15}.
%
Depuis $2017$, S. Darak et son équipe à l'IIIT Delhi en Inde, ont travaillé activement dans la recherche sur la radio intelligente à l'aide de bandits multi-bras.
Par héritage de son travail au sein de l'équipe SCEE,
certains de leurs travaux récents sont également illustrés par des démonstrations réalistes utilisant USRP et le système MATLAB/Simulink
\cite{KumarYadav2018,SawantKumar2018,JoshiKumar2018}.
%
Pour plus de détails sur l'état de la recherche sur la radio intelligente, nous renvoyons aux enquêtes \cite{garhwal2012survey,marinho2012cognitive}.


\paragraph{Limitations et spécificités des réseaux de l'IdO.}
%
La littérature susmentionnée a essentiellement montré que les algorithmes BMB peuvent être appliqués avec succès au problème de l'AOS.
Toutefois, si l'on considère des objets de RI qui ne peuvent pas effectuer de détection spectrale active, tels que des objets à faible coût et à faible consommation d'énergie conçus pour les futurs réseaux de l'IdO, le modèle BMB qui utilise cette détection active pour vérifier l'absence d'UP dans le cadre de l'AOS ne peut plus être appliqué.
En outre, dans la plupart des cas, les réseaux de l'IdO utilisent des bandes non licenciées et il n'y a donc plus de distinction entre UP et US.
En d'autres termes, il n'y a plus de priorité entre les utilisateurs des réseaux de l'IdO.
% (plus de UP, tous sont US).
%
Les autres spécificités des réseaux de l'IdO peuvent être énumérées comme suit,
et nous nous référons à l'article \cite{Centenaro16} pour plus de détails.
%
\begin{itemize}\tightlist
    % \item
    % nous considérons ici les réseaux de l'IdO qui fonctionnent dans des bandes sans licence (plus de distinction entre UP et US),
    \item
    La plupart des objets sont à faible coût, fonctionnent sur du matériel de mauvaise qualité et ont des capacités logicielles et d'alimentation limitées (c'est-à-dire de minuscules batteries).
    Cela signifie que même s'ils ne sont pas équipés de capacités de détection particulières,
    ils sont équipés d'un émetteur ainsi que d'un récepteur (Tx et Rx, \ie, un émetteur-récepteur),
    et ils ont des capacités de stockage et de calcul peu complexes (petits processeurs embarqués),
    \item
    une seule station de base IdO devra gérer un très grand nombre d'objets,
    et ne peut leur envoyer des ordres de coordination afin d'optimiser le réseau,
    \item
    les objets ont des cycles de fonctionnement faibles à très faibles (seulement quelques messages par minute ou par mois), et sont en charge d'initier leurs communications en voie montante (\emph{up-link}).
\end{itemize}

Une coordination centrale des objets avec leur station de base est donc impossible car il faudrait qu'ils communiquent en permanence avec la station de base, ce qui viole les deux dernières contraintes.
%
De plus, en raison de leur matériel limité (Rx/Tx), de leur puissance de calcul limitée et de la durée de vie limitée de leur batterie, bien que les objets soient capables d'utiliser leur antenne de réception pour écouter dans un canal occasionnellement (pendant quelques intervalles de temps après chaque message envoyé), ils ne peuvent pas effectuer de détection spectrale à chaque instant, comme cela est envisagé pour l'AOS (\ie, au début de chaque intervalle, dans un schéma de type \guillemotleft{} écoute avant de transmettre \guillemotright{}).

On pourrait penser qu'en l'absence de détection spectrale, l'apprentissage par renforcement n'est plus applicable, mais n'importe quel objet d'un vrai réseau de l'IdO reçoit encore des informations sur son environnement, après certaines ou toutes ses transmissions.
Dans la plupart des normes IdO, un message (sur la voie montante) envoyé à une station de base doit être suivi d'un message (sur la voie descendante) renvoyé par la station de base pour indiquer si le message a été bien reçu, décodé et compris.
%
En utilisant cette rétroaction, qui consiste en un \emph{acquittement} (ou \emph{acknowledgement}, \Ack) reçu peu après chaque transmission réussie, ou en l'absence d'\Ack{} après une transmission échouée, il est possible qu'un objet de l'IdO puisse également optimiser ses communications, grâce à un algorithme d'apprentissage par renforcement correctement conçu.



% ----------------------------------------------------------------------------
\section*{Nos contributions}

Nous commençons par formuler le problème étudié dans cette thèse, puis nous développons notre approche.
%
Si nous résumons les problèmes étudiés en une seule question, ce pourrait être la suivante :
\guillemotleft{} \emph{peut-on adapter les outils de prise de décision déjà appliqués avec succès à la Radio Intelligente (RI) pour l'Accès Opportuniste au Spectre (AOS) aux besoins spécifiques de la RI pour les (futurs) réseaux de l'Internet des Objets (IdO) ?} \guillemotright.
%
Nous répondons partiellement à cette problématique par les étapes de recherche suivantes.



\subsection*{Explorer la jungle des algorithmes BMB}

% \paragraph{Chapitre~\ref{chapter:2}.}
%
Nous avons commencé par explorer la riche littérature des bandits multi-bras,
car il existe de nombreux algorithmes différents, avec de nombreuses variantes du problème présenté ci-dessus.
Nous commençons donc la première partie de cette thèse par le Chapitre~\ref{chapter:2}, qui présente le modèle BMB et passe en revue les algorithmes les plus importants conçus pour résoudre les problèmes stochastiques et stationnaires de bandits.
%
Afin de bien comprendre quels algorithmes pourraient être adaptés aux contraintes susmentionnées de la RI pour les réseaux de l'IdO,
nous ne nous sommes pas seulement intéressés par la mesure habituelle des performances d'un algorithme BMB (\ie, son regret, voir ci-dessous en Section~\ref{sec:2:lowerUpperBoundsRegret}),
mais aussi par leurs performances empiriques en termes de complexité de calcul et de stockage, tant du point de vue des analyses théoriques que des mesures réelles.


% \paragraph{Chapitre~\ref{chapter:3}.}
%
Notre exploration du grand nombre d'algorithmes et de modèles BMB développés dans la littérature récente
nous a donné l'ambition d'écrire un seul logiciel permettant à n'importe quel-le chercheur-euse d'implanter facilement de nouveaux modèles et algorithmes, afin de comparer les modèles existants et d'explorer empiriquement les performances de nouveaux algorithmes.
Pour atteindre cet objectif, nous avons développé une bibliothèque de simulation des problèmes BMB, dans le langage populaire Python.
%
À notre connaissance, nous avons écrit la bibliothèque libre de simulation la plus exhaustive pour les problèmes BMB, appelée SMPyBandits \cite{SMPyBanditsJMLR,SMPyBandits}, qui est publiée en ligne sous licence libre (open-source) et qui est hébergée sur \href{https://GitHub.com/SMPyBandits}{\texttt{GitHub.com/SMPyBandits}}.
Nous présentons en détail son architecture et ses fonctionnalités dans le Chapitre~\ref{chapter:3}.
%  et différents exemples de son utilisation.
Une documentation complète est disponible en ligne, ainsi que des instructions pour reproduire les simulations présentées dans la suite de cette thèse.


% \paragraph{Chapitre~\ref{chapter:25}.}
%
Étant donné le grand nombre d'algorithmes BMB disponibles, nous nous intéressons aussi à la question de savoir comment un-e ingénieur-e peut choisir l'algorithme à mettre en œuvre, dans un objet de l'IdO donné, afin de doter cet objet de la capacité de s'adapter de manière robuste à son futur environnement.
Pour répondre à cette question, nous présentons deux approches.
La première est une comparaison empirique d'algorithmes existants, se focalisant sur les plus efficaces et les mieux connus, en Section~\ref{sec:3:reviewSPAlgorithms} et \ref{sec:3:timeAndMemoryCosts}.
Nous confirmons que les algorithmes les plus utilisés, tels que \UCB{} \cite{Auer02}, l'échantillonage de Thompson (Thompson sampling) \cite{Thompson33} et \klUCB{} \cite{KLUCBJournal}, sont les plus efficaces en termes de regret, et offrent un bon équilibre entre regret et complexité (temps et mémoire).
La seconde approche est la sélection d'algorithme en ligne, consistant à agréger un ensemble (fini) d'algorithmes et à découvrir automatiquement lequel est le plus efficace, contre un problème donné, avec notre contribution \Aggr{} que nous détaillons en Chapitre~\ref{chapter:25}.
Ce travail sur l'agrégation des algorithmes BMB était motivé par le cadre AOS, pour lequel les travaux antérieurs ne considéraient que l'algorithme $\UCB_1$ sans vraiment justifier ce choix.
Cette contribution a été présentée à la conférence IEEE WCNC à Barcelone, en Espagne, en avril 2018.


\subsection*{Nos modèles de réseaux de l'IdO et de BMB décentralisé}

Dans la deuxième partie de cette thèse, nous commençons par proposer et étudier différents modèles de réseaux de l'IdO, avec des simulations et un prototype réel (\ie, une preuve de concept), puis nous étudions des questions intéressantes sur deux modèles mathématiques de bandits, issues de la simplification de notre premier modèle.


% \paragraph{Chapitre~\ref{chapter:4}.}
%
Nous étudions deux modèles de réseaux de l'IdO dans le Chapitre~\ref{chapter:4}.
Le premier modèle considère que les objets ont simplement des données à envoyer régulièrement à une station de base, à des moments imprévisibles ou aléatoires.
De tels objets utilisent les acquittements comme un retour d'information, afin d'optimiser leurs communications sur la liaison montante, en accédant aux meilleurs canaux, c'est-à-dire le canal le moins occupé par le trafic ambiant.
L'objectif de ce premier modèle est d'améliorer la qualité de service (QdS, ou \emph{Quality of Service}, QoS) de l'application de ce réseau de l'IdO, en appliquant un ApR décentralisé (côté objet), afin de réduire le taux d'échec de transmission.
La station de base recevra donc plus de paquets de liaison montante des objets desservis, s'ils peuvent apprendre par eux-mêmes un accès efficace au spectre.
Cela implique aussi que le réseau peut accepter plus d'objets tout en maintenant la même QdS.
%
Notre deuxième modèle considère ensuite les retransmissions de paquets, et bien que les deux modèles soient similaires, l'application d'un algorithme d'apprentissage décentralisé efficace permet dans ce cas d'augmenter la durée de vie des batteries de chacun des objets, ainsi que la QdS de l'ensemble du réseau, puisque moins de retransmissions réduisent aussi la charge locale du spectre.

Le premier modèle sans retransmission est issu du premier article écrit au cours de cette thèse, qui a initié une collaboration avec R. Bonnefoi, un autre doctorant de notre équipe SCEE.
Nous l'avons présenté à la conférence EAI CROWNCOM à Lisbonne, au Portugal, en septembre $2017$, où il a obtenu le \guillemotleft{} prix du meilleur papier \guillemotright{} \cite{Bonnefoi17}.
%
La preuve de concept (PdC) a continué notre collaboration fructueuse, et a été montrée pendant trois jours à la conférence IEEE ICT à Saint-Malo, France, en juin $2018$ \cite{Besson2018ICT}, et aussi à la conférence IEEE WCNC à Marrakech, Maroc, en avril $2019$ \cite{Besson2019WCNC}.
%
Notre deuxième modèle, avec retransmissions, correspond à la dernière collaboration avec R. Bonnefoi, qui a été présentée à l'atelier MoTION, également pendant la conférence IEEE WCNC $2019$ \cite{Bonnefoi2019WCNC}.


% \paragraph{Chapitre~\ref{chapter:5}.}
%
Dans les deux modèles de réseaux de l'IdO présentés ci-dessus, l'idée maîtresse est de laisser chaque objet dynamique d'un tel réseau exécuter un algorithme d'apprentissage de manière totalement décentralisée, afin d'optimiser le système complet.
Il est important de noter que chaque objet ne communique, et donc n'apprend, qu'à quelques unes et non à toutes les étapes, en suivant son propre processus d'activation aléatoire (nous nous limitons à un processus d'activation purement aléatoire, Bernoulli de probabilité $p$).
Cela signifie que chaque objet vise son propre objectif local, qui est de maximiser sa récompense cumulée, d'une manière égoïste et complètement agnostique des autres objets ayant le même objectif.
Il est bien connu en théorie des jeux que jouer égoïstement peut être désastreux pour la mesure centralisée de la performance, on peut penser aux \guillemotleft{} dilemmes \guillemotright{} populaires, tels que le \href{https://fr.wikipedia.org/wiki/Dilemme_du_prisonnier}{dilemme du prisonnier}.
Il était donc assez surprenant que nos simulations numériques, ainsi que la preuve de concept réaliste, aient montré que l'apprentissage BMB décentralisé et égoïste conduisait à une coordination efficace entre les objets, dans tous les scénarios considérés, malgré le fait que ces objets ne peuvent communiquer directement entre eux, et reçoivent seulement un indicateur de collision depuis la station de base à laquelle ils sont associés.

Nous avons donc d'abord essayé d'analyser la performance de cet algorithme décentralisé, que nous appelons \Selfish, dans le modèle de la Section~\ref{sec:4:firstModel},
mais en raison du nombre aléatoire d'objets actifs à chaque pas de temps (\ie, dès que la probabilité $p$ d'activation est $p < 1$), nous n'avons pas pu développer une analyse propre.
%
C'est la raison pour laquelle nous affaiblissons l'hypothèse d'avoir $M \gg K$ (ou même simplement $M > K$) dans un réseau avec un canal sans fil orthogonal $K$ dans le Chapitre~\ref{chapter:5}, qui est équivalent à avoir $p < 1$, et nous considérons le cas d'objets actifs à chaque étape de temps (\ie, $p=1$), et nous nous limitons donc à $M \leq K$ objets.
Le modèle que nous avons étudié comporte différentes variantes, selon le niveau de rétroaction, et couvre à la fois le cas AOS (\ie, avec détection des UP) ou le cas IdO (\ie, sans détection).
L'objectif était de comprendre l'heuristique du Chapitre~\ref{chapter:4}, \Selfish, dans le cadre plus simple des BMB multi-joueurs, qui a été étudié auparavant pour le cas de l'AOS, comme étudié par exemple par \cite{Zhao10,Anandkumar10,Anandkumar11}.
%
Le cas AOS couvert par notre modèle peut sembler légèrement différent de celui considéré par \cite{Jouini10} et d'autres travaux antérieurs,
car notre modèle exige qu'un \Ack{} soit renvoyé par la station de base si la transmission a réussi, même dans le cas où l'information de détection est disponible.
Il est en fait équivalent aux modèles précédemment étudiés d'applications des BMB pour l'AOS, puisqu'ils ne prennent en compte que des objets synchronisés, un US et des UP, et donc si la détection indique qu'un canal est libre pour un intervalle de temps, le message envoyé par l'US sera certainement reçu avec succès par la station de base (dans le modèle idéal), donc il n'y aura aucun risque de collision, donc il n'a pas besoin d'\Ack.


D'une part, dans le cadre de l'AOS, nous n'avons pas réussi à obtenir de résultat positif pour l'algorithme \Selfish, car nous avons prouvé que dans certains petits problèmes (\eg, $K=3$ et $M=2$), \Selfish-\UCB{} peut présenter un regret linéaire avec une faible probabilité, et donc souffre d'un regret moyen linéaire.
Ce résultat a ensuite été confirmé et analysé par d'autres chercheurs dans \cite{BoursierPerchet18} (Annexe~F).
%
D'autre part, nous avons été en mesure de proposer de nouveaux algorithmes pour ce problème de bandit multi-joueurs (avec informations de détection), et nous avons analysé notre proposition, appelée \MCTopM, pour montrer qu'elle atteint une borne supérieure de regret logarithmique à temps fini, qui améliore considérablement les résultats précédents.
Notre algorithme atteint également un nombre logarithmique de collisions et de changements de bras, et permet à un groupe fixe de $M$ objets d'apprendre efficacement à utiliser les $M$ meilleurs canaux orthogonalement pour presque toutes leurs communications montantes.
%
Ce fort résultat théorique dépend fortement de la présence d'une information de détection spectrale et, par conséquent, nos résultats ne sont pas (encore) applicables au modèle IdO sans détection.


Comme expliqué plus haut, le modèle de Chapitre~\ref{chapter:4} s'est révélé trop complexe à analyser, principalement parce que nous considérons un réseau d'IdO avec de nombreux objets, chacun suivant des profils d'activation aléatoires.
La difficulté ne réside pas dans le fait que nous essayons d'analyser des algorithmes de bandits, conçus pour résoudre des problèmes stationnaires, appliqués à un problème non stationnaire,
mais plutôt que nous considérons des algorithmes qui jouent tous à différents instants d'activations (aléatoires), ce qui donne des sous-ensembles différents et imprédictibles des pas de temps synchronisés globaux.
Généraliser à différentes probabilités d'activation conduirait à un modèle encore plus difficile à analyser, et imposer un maximum de $M \leq K$ objets n'est en fait pas vraiment réaliste pour les réseaux de l'IdO, même si cela conduit au modèle du Chapitre~\ref{chapter:5}.

D'une part, si le profil d'activation de tous les périphériques peut être corrigé, par exemple en se basant sur une affectation centralisée des périphériques à différents créneaux horaires, alors le modèle de bandit multi-joueurs du Chapitre~\ref{chapter:5} peut être utilisé pour permettre à chaque groupe de $M \leq K$ périphériques d'apprendre une affectation orthogonale optimale dans les $K$ canaux (\eg, les groupes peuvent être le jeu des périphériques que tous émettent au même instant).
%
Cependant, même si nous continuons à supposer un temps synchronisé dans tout le reste de la thèse,
il est difficile de soutenir que cette hypothèse d'un calendrier centralisé pour les objets peut être réaliste, car nous étudions le cas de l'apprentissage décentralisé pour l'AOS et l'IdO précisément afin d'éviter un surcoût dû à une signalisation régulière et un contrôle central des objets par la station de base, comme nous l'avons expliqué précédemment.

D'autre part, une autre façon d'affaiblir l'hypothèse de non stationnarité est de prendre le point de vue d'un seul objet, comme dans le cas de l'AOS mentionné ci-dessus, où l'accent est mis sur un US entouré de plusieurs UP ayant des comportements stochastiques et stationnaires.
Si nous nous concentrons sur un objet de l'IdO, son environnement (\ie, les autres objets) est non stationnaire, ce qui signifie que ses propriétés moyennes peuvent fluctuer dans le temps, mais les environnements réels présentent généralement des non-stationnarité avec une certaine structure.
C'est pourquoi nous considérons l'hypothèse de stationnarité par intervalles de temps, dans la dernière contribution et le dernier chapitre de cette thèse.


% \paragraph{Chapitre~\ref{chapter:6}.}
%
Afin de bien comprendre également comment les algorithmes BMB se comportent dans un environnement non stationnaire, nous avons étudié la littérature sur les BMB adverses ou non stationnaires, principalement pour les deux cas d'environnements variant lentement ou changeant brusquement.
Nous commençons notre dernier Chapitre~\ref{chapter:6} en passant en revue les travaux existants,
et nous avons choisi de nous concentrer sur les problèmes de bandits stationnaires par morceaux,
ce qui signifie que le problème de bandit sous-jacent est stationnaire sur certains intervalles, séparés par des points de changement situés à des moments inconnus.
Supposer que l'environnement est stationnaire par morceaux est en effet cohérent pour les applications aux réseaux sans fil, où un point de changement peut par exemple correspondre à l'arrivée d'un nouveau groupe d'objets dans le réseau. Des exemples d'une telle situation peuvent être une nouvelle entreprise arrivant sur le marché dans une ville (par exemple, les compteurs Linky qui sont installés aujourd'hui en France), ou l'agriculteur voisin installant des capteurs sur son propre troupeau de vaches, etc.
%
Deux grandes familles d'algorithmes ont été proposées pour les problèmes stationnaires par morceaux,
qui consistent généralement à combiner un algorithme efficace conçu pour le problème des bandits stationnaires (\eg, Thompson Sampling ou \klUCB), avec un moyen de s'adapter aux changements dans les distributions des bras.
Les algorithmes passivement adaptatives utilisent une fenêtre de taille fixe ou évolutive \cite{Garivier11UCBDiscount}, ou un facteur d'oubli, pour oublier les observations passées \cite{Kocsis06,Gupta11thompson},
tandis que les algorithmes activement adaptatifs utilisent un test statistique pour détecter les points de changement \cite{MellorShapiro13,Allesiardo15}.
%
Comme la littérature récente a montré que cette dernière approche est généralement plus compétitive, en obtenant de meilleurs résultats tant sur le plan empirique que théorique, nous avons choisi de développer notre propre algorithme activement adaptatif.

Suite à deux travaux précédents récents \cite{LiuLeeShroff17,CaoZhenKvetonXie18}, nous combinons une stratégie d'indices efficace (\klUCB) et un test efficace de détection des points de changement, dans l'hypothèse de récompenses bornées.
Nous nous appuyons sur des résultats très récents à propos du Test du Rapport de Vraisemblance Généralisé (\emph{Generalized Likelihood Ratio Test}, GLRT) pour les variables gaussiennes et sous-gaussiennes \cite{Maillard2018GLR}, mais nous nous concentrons plutôt sur les récompenses bornées et les distributions de Bernoulli.
Les récompenses bornées sont en effet généralement plus appropriées pour les applications de RI, comme nous l'avons discuté ci-dessus.
En utilisant le fait que les variables bornées dans $[0,1]$ ne sont pas seulement $1/4$ sous-Gaussiennes mais aussi sous-Bernoulli, nous prouvons les premières garanties à temps fini pour le GLRT pour des variables bornées.
Nous montrons d'abord des bornes à temps finis à la fois sur la probabilité de fausse alarme de notre test, et sur son délai de détection, sous des hypothèses raisonnables sur la longueur des séquences stationnaires.
Notre algorithme est alors présenté en deux variantes, selon que les points de changement soient locaux (\ie, une seule moyenne de bras change à chaque point de changement) ou globaux (\ie, peut-être que toutes les moyennes des bras changent à la fois).
%
Nous prouvons qu'en combinant l'algorithme \klUCB, asymptotiquement optimale pour les problèmes stationnaires, et notre nouvelle analyse du GLRT pour les récompenses bornées, nous obtenons des garanties de pointe sur le regret de notre algorithme proposé, noté \GLRklUCB.
Nous considérons la même hypothèse que nos concurrents, en supposant que la longueur des intervalles stationnaires est \guillemotleft{} assez longue \guillemotright{} pour que les points de changement soient détectables.
La meilleure borne supérieure de regret est obtenue lorsque l'algorithme connaît à l'avance l'horizon et le nombre de points de changement, mais notre algorithme n'a pas besoin d'avoir d'autres connaissances sur la difficulté du problème pour être paramétré de manière optimale.
%
La performance de \GLRklUCB{} est également illustrée par des expériences numériques sur des données synthétiques, où il est démontré qu'il surpasse tous les algorithmes passivement adaptatifs ainsi que les précédents algorithmes activement adaptatifs.
%
Ce dernier chapitre est basé sur un de nos derniers travaux, publié à la conférence GRETSI à Lille, en août $2019$ \cite{Besson2019Gretsi}.
Notre travail a aussi mené à une version longue \cite{Besson2019GLRT}, qui sera complétée avec des résultats plus récents, afin de le soumettre bientôt à une revue (probablement le Journal of Machine Learning Research) avant décembre $2019$.


% \paragraph{Annexe.}
%
Ce manuscrit se termine par
une liste d'abréviations et de notations, puis de figures, d'algorithmes, de morceaux de code et de tableaux,
et enfin des références bibliographiques.


% FIXME VRAIMENT BEAUCOUP raccourcir ce qui précéde !


% ----------------------------------------------------------------------------
\section*{Contributions}

Nous pouvons énumérer les points suivants pour résumer les principales contributions de cette thèse.
Elles se situent à trois niveaux : des modèles BMB pour plusieurs contextes radio liés à la radio intelligente et l'IdO ; des preuves théoriques associées aux problèmes d'apprentissages correspondants ; et enfin des implantations d'algorithmes à but de simulations, de partage avec la communauté, et de validation par des démonstrations radio réalistes.

Nous signalons le chapitre, ainsi que les conférences nationales ou internationales dans lesquelles ont été publiés les résultats correspondants.

\begin{itemize}
    % \item
    % Nous présentons les concepts et les notations du modèle des bandits multi-bras, dans le chapitre~\ref{chapter:2} selon un point de vue mathématique formel.
    % Nous suivons aussi une approche didactique, puisque nous utilisons en Section~\ref{par:2:interactiveDemoDiscoverMAB} une démonstration interactive en ligne conçue pour permettre à n'importe qui de jouer contre un petit problème de bandit depuis son navigateur.

    % \item
    % Nous donnons une brève revue de la littérature de recherche concernant les algorithmes stochastiques de bandits en Section~\ref{sec:2:famousMABalgorithms}.

    \item
    Nous avons écrit la bibliothèque de simulation pour les problèmes BMB la plus complète, appelée SMPyBandits, qui est écrite en Python et publiée en ligne sous une licence open-source \cite{SMPyBanditsJMLR,SMPyBandits}.
    Nous présentons en détail son architecture et ses fonctionnalités dans le Chapitre~\ref{chapter:3}, ainsi que différents exemples de son utilisation.
    Une documentation complète est disponible en ligne, ainsi que des instructions exhaustives pour reproduire les simulations numériques utilisées tout au long de ce manuscrit de thèse.

    \item
    Nous présentons le problème du choix de l'algorithme qu'un-e ingénieur-e devrait utiliser, ou de la sélection d'algorithme parmi la riche collection des différents algorithmes de bandits dans la littérature, dans le Chapitre~\ref{chapter:25}.
    Nous présentons un algorithme d'agrégation d'algorithmes, appelé \Aggr, comme une solution en ligne au problème de sélection d'algorithmes, et nous montrons par des simulations numériques qu'il peut atteindre de bonnes performances empiriques
    (\cite{Besson2018WCNC}, publié à WCNC 2018).

    \item
    Nous proposons différents modèles pour les réseaux de l'Internet des Objets (IdO), dans le Chapitre~\ref{chapter:4}, où les objets dotés de capacités de radio intelligente peuvent mettre en œuvre de leur côté des algorithmes BMB, pour augmenter automatiquement la durée de vie de leur batterie.
    Cela permet également à davantage d'objets d'utiliser le même réseau tout en maintenant un taux élevé d'accès au spectre sans souffrir de collisions radio
    (publiés à CROWNCOM 2017 \cite{Bonnefoi17}, démonstration ICT 2018 \cite{Besson2018ICT}, WCNC 2019 \cite{Besson2019WCNC} et MOTIoN 2019 \cite{Bonnefoi2019WCNC}).

    \item
    Nous avons réalisé une démonstration réaliste du modèle proposé ci-dessus \cite{Besson2018ICT,Besson2019WCNC}, présentée à ICT 2018, et nous la détaillons en Section~\ref{sec:4:gnuradio}. Nous avons aussi réalisé une vidéo de présentation de six minutes, hébergée sur \texttt{\href{https://youtu.be/HospLNQhcMk}{youtu.be/HospLNQhcMk}}.

    % \item
    % Le code source des deux contributions susmentionnées est publié en ligne, ainsi que des instructions pour la reproduction de nos travaux (avec GNU Octave ou MATLAB).

    \item
    Nous formalisons le modèle du bandit multi-joueur, pour lequel nous avons introduit deux variantes, dans le Chapitre~\ref{chapter:5}.
    Pour le cas avec information de détection spectrale (\emph{spectral sensing}), nous proposons deux algorithmes, et nous analysons notre algorithme \MCTopM, pour prouver qu'il est très efficace, à temps fini et asymptotiquement.
    Nous présentons aussi des expériences numériques approfondies pour montrer qu'il est bien plus efficace que les autres algorithmes de la littérature.
    Notre travail \cite{Besson2018ALT}, publié à ALT 2018, a également contribué à un nouvel élan à la recherche sur les bandits multi-joueurs, certains travaux de recherche récents s'étant construits sur nos résultats.

    \item
    Nous donnons une revue détaillée de la littérature sur les différentes extensions du modèle BMB multi-joueurs, particulièrement active ces deux dernières années.
    % qui, selon nous, n'a jamais été écrit auparavant.
    % (ALT 2018, et 4-8 travaux inspirés de notre article depuis lors). Etat de l'art avec notre algorithme MCTopM + klUCB pour les bandits multi-joueurs ``avec détection'', état de l'art empirique avec notre approche simple (mais erronée) ``égoïste'' en cas de ``non détection''.

    \item
    Nous présentons ensuite le modèle de bandits multi-bras stationnaire par morceaux, dans le Chapitre~\ref{chapter:6}, et une vue détaillée de l'état de l'art de la recherche sur ce modèle (\cite{Besson2019Gretsi} publié à GRETSI 2019, \cite{Besson2019GLRT}).
    Nous proposons un nouvel algorithme activement adaptatif, pour ces problèmes stationnaires par morceaux, \GLRklUCB, qui atteint des performances empiriques et des garanties théoriques comparables à l'état de l'art, tout en utilisant des hypothèses plus faibles car notre approche n'a pas besoin de connaître de bornes sur la difficulté du problème à part le nombre de points de ruptures.
    % Revue de littérature sur les modèles et algorithmes non stationnaires, état de l'art de la station stationnaire à la pièce avec notre algorithme, test GLR + klUCB

    % \item
    % La dernière contribution de cette thèse est notre article \cite{Besson2018DoublingTricks}.
    % % , résumé en Annexe~\ref{app:2:DoublingTricks}.
    % Cet article donne une revue de la littérature sur l'utilisation de la technique de doublements successifs de l'horizon (``doubling trick'') pour les problèmes de bandits,
    % ainsi qu'une analyse unifiée et plus générique de deux familles de doublements.
\end{itemize}

% ----------------------------------------------------------------------------
\section*{Organisation du manuscrit}

% Ce manuscrit est organisé comme suit.
%
L'ordre de lecture du manuscrit peut suivre n'importe quel chemin, entre l'introduction donnée dans le Chapitre~\ref{chapter:1}, et la conclusion générale qui constitue le dernier Chapitre~\ref{chapter:conclusion}.
Comme le montre le graphique de la Figure~\ref{fig:1:organization_fr} ci-dessous,
la thèse est organisée en deux parties, correspondant aux deux lignes intermédiaires de la figure suivante.

\begin{figure}[h!]
    \centering
    \resizebox{0.95\textwidth}{!}{
    \begin{tikzpicture}[>=latex',line join=bevel,scale=2.25]
        %
        \node[align=center] (introduction) at (0,3.25) [rectangle,draw,fill=blue!15] {\textbf{Chapitre~\ref{chapter:1}}\\Introduction};
        \node[align=center] (chapter2) at (0,2.25) [rectangle,draw,fill=red!15] {\textbf{Chapitre~\ref{chapter:2}}\\Modèle bandits multi-bras\\stochastique et stationnaire};
        \node[align=center] (chapter3) at (-2.5,2.25) [rectangle,draw,fill=red!10] {\textbf{Chapitre~\ref{chapter:3}}\\SMPyBandits : bibliothèque\\de simulations pour BMB};
        \node[align=center] (chapter25) at (2.5,2.25) [rectangle,draw,fill=red!20] {\textbf{Chapitre~\ref{chapter:25}}\\Sélection séquentielle\\d'algorithme BMB};
        \node[align=center] (chapter4) at (-2.5,1) [rectangle,draw,fill=green!10] {\textbf{Chapitre~\ref{chapter:4}}\\Deux modèles BMB\\pour les réseaux de l'IdO};
        \node[align=center] (chapter5) at (0,1) [rectangle,draw,fill=green!15] {\textbf{Chapitre~\ref{chapter:5}}\\Modèle BMB\\Multi-joueurs};
        \node[align=center] (chapter6) at (2.5,1) [rectangle,draw,fill=green!20] {\textbf{Chapitre~\ref{chapter:6}}\\Modèle BMB\\Non-stationnaire};
        \node[align=center] (conclusion) at (0,-0.25) [rectangle,draw,fill=blue!20] {\textbf{Chapitre~\ref{chapter:conclusion}}\\Conclusion générale};
        % \node[align=center] (appendix) at (2.5,-0.25) [rectangle,draw,fill=yellow!10] {Annexes};
        %
        \draw [color=black,thick,->] (introduction) to (chapter2);
        \draw [color=black,thick,<->] (chapter2) to (chapter3);
        \draw [color=black,thick,<->] (chapter2) to (chapter25);
        \draw [color=black,thick,->] (chapter2) to (chapter4);
        \draw [color=black,thick,->] (chapter2) to (chapter5);
        \draw [color=black,densely dotted,<->]   (chapter4) to (chapter5);
        % \draw [color=black,densely dotted,->] -| (chapter3) to (chapter6);
        % \draw [color=black,densely dotted,->] -| (chapter25) to (chapter5);
        \draw [color=black,densely dotted,<->]   (chapter5) to (chapter6);
        \draw [color=black,thick,->] (chapter2) to (chapter6);
        \draw [color=black,thick,->] (chapter4) to (conclusion);
        \draw [color=black,thick,->] (chapter5) to (conclusion);
        \draw [color=black,thick,->] (chapter6) to (conclusion);
        % \draw [color=black,thick,->] (conclusion) to (appendix);
        %
    \end{tikzpicture}
    }
    \caption[Organisation de la thèse : une carte de lecture.]{Organisation de la thèse. Cette thèse peut se lire en suivant n'importe quel chemin contenant le Chapitre~\ref{chapter:1}, le Chapitre~\ref{chapter:2}, au moins un des trois Chapitres~\ref{chapter:4}, \ref{chapter:5} ou \ref{chapter:6}, et la Conclusion.}
    \label{fig:1:organization_fr}
\end{figure}

\begin{itemize}
    \item
\textcolor{darkred}{Dans la Partie~\ref{part:Introduction}}, nous commençons par le Chapitre~\ref{chapter:2} qui présente les modèles de bandits multi-bras (BMB), les concepts et les notations utilisés dans tout ce document. Ce premier chapitre est nécessaire à la lecture du reste du manuscrit.
Le Chapitre~\ref{chapter:3} présente notre bibliothèque de simulations SMPyBandits, qui est utilisée par les Chapitres~\ref{chapter:2}, \ref{chapter:25}, \ref{chapter:5} et \ref{chapter:6} pour leurs simulations numériques.
%  lire ce chapitre n'est pas obligatoire pour comprendre la suite du document.
Nous terminons cette première partie par le Chapitre~\ref{chapter:25},
% qui n'est pas non plus nécessaire à la compréhension globale du manuscrit, mais
qui détaille la première contribution : un nouvel algorithme pour la sélection séquentielle d'algorithmes BMB.

    \item
\textcolor{darkgreen}{La deuxième Partie~\ref{part:MABIOT}} contient ensuite trois chapitres, qui sont inclus à la fois dans l'ordre logique et chronologique, mais peuvent être lus quasiment indépendamment.
Le Chapitre~\ref{chapter:4} commence par proposer et étudier différents modèles de réseaux de l'IdO, pour lesquels nous montrons que les algorithmes BMB peuvent être utilisés avec succès. Nos deux modèles sont intéressants et proches de la réalité, mais ils se sont révélés trop complexes pour proposer une analyse mathématique de la bonne performance empirique des solutions envisagées.
Pour cette raison, nous simplifions les modèles précédents dans la suite du document,
afin d'établir des preuves mathématiques garantissant la convergence, ainsi que les gains en performance apportés par ces algorithmes de bandit.
Les deux Chapitres~\ref{chapter:5} et \ref{chapter:6} étudient chacun un modèle intermédiaire, situé entre le modèle BMB stationnaire à un joueur du Chapitre~\ref{chapter:2} et les modèles de réseaux de l'IdO du Chapitre~\ref{chapter:4}.
Ces deux études ont chacune donné des résultats théoriques à la pointe de la recherche, sur les deux modèles de bandits multi-joueurs et de bandits stationnaires par morceaux, que nous avons aussi validé par des expériences numériques.

\end{itemize}


% WARNING
\vfill{}

\hr{}

\textbf{Note sur le droit intellectuel.}
%
Ce document et les ressources additionnelles requises pour le générer (notamment les fichiers \LaTeX, les morceaux de code Python, les figures etc)
sont \href{https://github.com/Naereen/phd-thesis/}{distribuées publiquement},
selon les termes de la \href{https://lbesson.mit-license.org/}{\emph{licence MIT}} open-source,
en ligne sur \href{https://github.com/Naereen/phd-thesis/}{\texttt{GitHub.com/Naereen/phd-thesis/}}.

\TODOL{Open source the repository as soon as I defended my thesis!}

\begin{center}
    \textbf{Copyright 2016-2019, \copyright ~Lilian~Besson.}
\end{center}


\end{resume_fr}
