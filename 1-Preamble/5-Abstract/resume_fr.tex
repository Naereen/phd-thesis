\begin{resume_fr}

% Résumé de la thèse en français.

\TODOL{Je ne sais pas quelle longueur ni quel niveau de détail il faut viser ici. Est-ce que ces 3/4 pages suffisent ? Est-ce qu'il faut donner des maths en français pour CHAQUE CHAPITRE ? J'ai peur de ne pas réussir à écrire 20 pages avec des maths en français juste pour pouvoir citer tel théorème ou tel résultat.}

\TODOL{En fait, maintenant que le ch1 est écrit, je peux traduire (et couper un peu) ce chapitre 1, et ça ira !}

% ----------------------------------------------------------------------------

% Chapitre 1
Dans cette thèse de doctorat, nous étudions les réseaux sans fil et les appareils reconfigurables qui peuvent accéder à des réseaux de radio intelligente, dans des bandes non licenciées et sans supervision centrale.
Plus spécifiquement, nous considérons des réseaux de l'Internet des Objets (IoT), avec l'objectif d'augmenter la durée de vie de la batterie des appareils, en les équipant d'algorithmes d'apprentissage machine peu coûteux mais efficaces, qui leur permettent d'améliorer automatiquement l'efficacité de leurs communications sans fil (Chapitre~\ref{chapter:1}).
% Chapitre 4
Nous proposons différents modèles de réseaux IoT, et nous montrons empiriquement, par des simulations numériques et une validation expérimentale réaliste, le gain que peuvent apporter nos méthodes, qui utilisent l'apprentissage par renforcement (Chapitre~\ref{chapter:4}).
% Chapitre 2
Les différents problèmes d'accès au réseau sont modélisés avec des Bandits Multi-Bras (MAB, Chapitre~\ref{chapter:2}), mais leur analyse est difficile à réaliser,
% , Chapitre~\ref{chapter:2}
car il est difficile de prouver la convergence de nombreux appareils jouant à un jeu collaboratif sans communication aucune coordination, lorsque les appareils suivent tous un modèle d'activation aléatoire.
% En effet, même le nombre de périphériques actifs à chaque instant est hautement imprévisible, rendant l'environnement non pas stationnaire et évoluant rapidement.
% Chapitres 5 et 6
Le reste de ce manuscrit étudie donc deux modèles restreints, d'abord des bandits multi-joueurs dans des problèmes stationnaires (Chapitre~\ref{chapter:5}), puis des bandits mono-joueurs non stationnaires (Chapitre~\ref{chapter:6}).
% Chapitre 3
Nous détaillons également une autre contribution, la bibliothèque Python open-source SMPyBandits pour des simulations numériques de problèmes MAB, qui couvre tous les modèles étudiés et d'autres (Chapitre~\ref{chapter:3}).


% ----------------------------------------------------------------------------
\section*{Contributions}

Nous pouvons énumérer les points suivants pour résumer les principales contributions de cette thèse.
Nous signalons les conférences internationales auxquelles ont été publiées certains résultats.

\begin{itemize}
    % \item
    % Nous présentons les concepts et les notations du modèle des bandits multi-bras, dans le chapitre~\ref{chapter:2} selon un point de vue mathématique formel.
    % Nous suivons aussi une approche didactique, puisque nous utilisons en Section~\ref{par:2:interactiveDemoDiscoverMAB} une démonstration interactive en ligne conçue pour permettre à n'importe qui de jouer contre un petit problème de bandit depuis son navigateur.

    % \item
    % Nous donnons une brève revue de la littérature de recherche concernant les algorithmes stochastiques de bandits en Section~\ref{sec:2:famousMABalgorithms}.

    \item
    Nous avons écrit la plus complète bibliothèque de simulation pour les problèmes MAB, appelée SMPyBandits, qui est écrite en Python et publiée en ligne sous une licence open-source \cite{SMPyBandits,SMPyBanditsJMLR}.
    Nous présentons en détail son architecture et ses fonctionnalités dans le Chapitre~\ref{chapter:3}, ainsi que différents exemples de son utilisation.
    Une documentation complète est disponible en ligne, ainsi que des instructions exhaustives pour reproduire les expériences utilisées dans la suite de cette thèse.

    \item
    Nous présentons le problème du choix de l'algorithme qu'un praticien devrait utiliser, ou de la sélection d'algorithme parmi la riche collection de différents algorithms MAB disponibles, dans le Chapitre~\ref{chapter:25}.
    Nous présentons un algorithme appelé \Aggr{} pour l'agrégation d'algorithmes, comme une solution en ligne au problème de sélection d'algorithmes, et nous montrons par quelques simulations numériques qu'il atteint de bonnes performances empiriques
    (WCNC 2018) \cite{Besson2018WCNC}.

    \item
    Nous proposons différents modèles pour les réseaux IdO (IoT), dans le Chapitre~\ref{chapter:4}, où les appareils dotés de capacités de radio intelligente peuvent implémenter de leur côté des algorithmes MAB, pour augmenter automatiquement la durée de vie de leur batterie. Cela permet à davantage de périphériques d'utiliser le même réseau tout en maintenant une haute qualité de service
    (CROWNCOM 2017, ICT demo 2018, WCNC 2019 et MOTIoN 2019)
    \cite{Bonnefoi17,Besson2019WCNC,Bonnefoi2019WCNC}.

    \item
    Nous avons implémenté une preuve de concept du modèle susmentionné \cite{Besson2018ICT}, et nous la présentons en détail en Section~\ref{sec:4:gnuradio}. Nous avons aussi réalisé une courte vidéo présentant notre démonstration, hébergée sur \texttt{\href{https://youtu.be/HospLNQhcMk}{youtu.be/HospLNQhcMk}}.

    % \item
    % Le code source des deux contributions susmentionnées est publié en ligne, ainsi que des instructions pour la reproduction de nos travaux (avec GNU Octave ou MATLAB).

    \item
    Nous formalisons le modèle du bandit multi-joueur, pour lequel nous avons introduit trois variantes, dans le Chapitre~\ref{chapter:5}.
    Pour le cas avec information de détection (``sensing''), nous proposons deux nouveaux algorithmes, et nous donnons une analyse pour notre algorithme \MCTopM, qui prouve qu'il est asymptotiquement optimal.
    Nous présentons aussi des expériences numériques approfondies pour montrer qu'il est bien plus efficace que les autres algorithmes de la littérature précédente.
    Notre travail \cite{Besson2018ALT} (ALT 2018) a également contribué à une nouvelle impulsion à la recherche sur les bandits multi-joueurs, car certains travaux de recherche récents se sont construits sur nos résultats.

    \item
    Nous donnons également une revue détaillée de la littérature sur les différentes extensions du modèle MAB multi-joueurs, qui, selon nous, n'a jamais été écrit auparavant.
    % (ALT 2018, et 4-8 travaux inspirés de notre article depuis lors). Etat de l'art avec notre algorithme MCTopM + klUCB pour les bandits multi-joueurs "avec détection", état de l'art empirique avec notre approche simple (mais erronée) "égoïste" en cas de "non détection".

    \item
    Nous présentons ensuite le modèle de bandits multi-bras stationnaire par morceaux, dans le chapitre~\ref{chapter:6}, et une vue détaillée de l'état de l'art de la recherche sur ce modèle \cite{Besson2019GLRT,Besson2019Gretsi} (GRETSI 2019).
    Suite à deux travaux récents, nous proposons un nouvel algorithme activement adaptatif pour ce problème stationnaire par morceaux, \GLRklUCB, qui atteint des performances comparables à l'état de l'art.
    % Revue de littérature sur les modèles et algorithmes non stationnaires, état de l'art de la station stationnaire à la pièce avec notre algorithme, test GLR + klUCB

    \item
    La dernière contribution de cette thèse est une revue de la littérature sur les cas d'utilisation possibles de la technique du doublement successif de l'horizon (``doubling trick'') pour les problèmes de bandits,
    ainsi qu'une analyse unifiée et plus générique de deux familles de doublements.
    Ceci a conduit à l'article \cite{Besson2018DoublingTricks}, qui est rapidement présenté en Annexe~\ref{app:2:DoublingTricks}.
\end{itemize}

% ----------------------------------------------------------------------------
\section*{Organisation du manuscrit}

Ce manuscrit est organisé comme suit.

\begin{figure}[h!]
    \centering
    \resizebox{0.95\textwidth}{!}{
    \begin{tikzpicture}[>=latex',line join=bevel,scale=2.25]
        %
        \node[align=center] (introduction) at (0,3.25) [rectangle,draw,fill=blue!10] {\textbf{Chapitre~\ref{chapter:1}}\\Introduction};
        \node[align=center] (chapter2) at (0,2.25) [rectangle,draw,fill=red!10] {\textbf{Chapitre~\ref{chapter:2}}\\Modèle bandits multi-bras\\stochastique et stationnaire};
        \node[align=center] (chapter3) at (-2.5,2.25) [rectangle,draw,fill=red!10] {\textbf{Chapitre~\ref{chapter:3}}\\SMPyBandits : bibliothèque\\de simulations pour MAB};
        \node[align=center] (chapter25) at (2.5,2.25) [rectangle,draw,fill=red!10] {\textbf{Chapitre~\ref{chapter:25}}\\Sélection séquentielle\\d'algorithme MAB};
        \node[align=center] (chapter4) at (-2.5,1) [rectangle,draw,fill=green!10] {\textbf{Chapitre~\ref{chapter:4}}\\Deux modèles MAB\\pour les réseaux IoT};
        \node[align=center] (chapter5) at (0,1) [rectangle,draw,fill=green!10] {\textbf{Chapitre~\ref{chapter:5}}\\Modèle MAB\\Multi-joueurs};
        \node[align=center] (chapter6) at (2.5,1) [rectangle,draw,fill=green!10] {\textbf{Chapitre~\ref{chapter:6}}\\Modèle MAB\\Non-stationnaire};
        \node[align=center] (conclusion) at (0,-0.25) [rectangle,draw,fill=blue!10] {\textbf{Chapitre~\ref{chapter:conclusion}}\\Conclusion générale};
        \node[align=center] (appendix) at (2.5,-0.25) [rectangle,draw,fill=blue!10] {Annexes};
        %
        \draw [color=black,thick,->] (introduction) to (chapter2);
        \draw [color=black,thick,<->] (chapter2) to (chapter3);
        \draw [color=black,thick,<->] (chapter2) to (chapter25);
        \draw [color=black,thick,->] (chapter2) to (chapter4);
        \draw [color=black,thick,->] (chapter2) to (chapter5);
        \draw [color=black,densely dotted,<->]   (chapter4) to (chapter5);
        % \draw [color=black,densely dotted,->] -| (chapter3) to (chapter6);
        % \draw [color=black,densely dotted,->] -| (chapter25) to (chapter5);
        \draw [color=black,densely dotted,<->]   (chapter5) to (chapter6);
        \draw [color=black,thick,->] (chapter2) to (chapter6);
        \draw [color=black,thick,->] (chapter4) to (conclusion);
        \draw [color=black,thick,->] (chapter5) to (conclusion);
        \draw [color=black,thick,->] (chapter6) to (conclusion);
        \draw [color=black,thick,->] (conclusion) to (appendix);
        %
    \end{tikzpicture}
    }
    \caption[Organisation de la thèse : une carte de lecture]{Organisation de la thèse. Cette thèse peut se lire en suivant n'importe quel chemin contenant le Chapitre~\ref{chapter:1}, le Chapitre~\ref{chapter:2}, au moins un des trois Chapitres~\ref{chapter:4}, \ref{chapter:5} et \ref{chapter:6}, et la Conclusion.}
    \label{fig:1:organization_fr}
\end{figure}

L'ordre de lecture du manuscrit peut être n'importe quel chemin, descendant entre l'introduction donnée dans Chapitre~\ref{chapter:1}, et la conclusion générale qui constitue le dernier Chapitre~\ref{chapter:conclusion}.
Comme le montre le graphique de la Figure~\ref{fig:1:organization_fr} ci-dessus,
la thèse est organisée en deux parties :

\begin{itemize}
    \item
\textcolor{darkred}{Dans la Partie~\ref{part:Introduction}}, nous commençons par le Chapitre~\ref{chapter:2} où nous présentons les modèles MAB, les concepts et les notations utilisés dans tout ce document. Ce premier chapitre est nécessaire à la lecture du reste du manuscrit.
Par contre le Chapitre~\ref{chapter:3} présente notre bibliothèque de simulations SMPyBandits, et même si les Chapitres~\ref{chapter:2}, \ref{chapter:25}, \ref{chapter:5} et \ref{chapter:6} utilisent la bibliothèque pour leurs simulations numériques, lire ce chapitre n'est pas obligatoire pour comprendre la suite du document.
Nous terminons cette première partie par le Chapitre~\ref{chapter:25}, qui n'est pas non plus nécessaire à la compréhension de la suite du manuscrit, et qui détaille la première contribution : un nouvel algorithme pour la sélection séquentielle d'algorithmes MAB.

    \item
\textcolor{darkgreen}{La deuxième Partie~\ref{part:MABIOT}} contient ensuite trois chapitres, qui sont inclus à la fois dans l'ordre logique et chronologique, mais peuvent être lus presque indépendamment.
Le Chapitre~\ref{chapter:4} commence par présenter différents modèles de réseaux IdO où les algorithmes MAB ont été utilisés avec succès. Nos deux modèles sont intéressants et proches de la réalité, mais ils se sont révélés trop généraux pour proposer une analyse mathématique de la bonne performance empirique des solutions envisagées.
Pour cette raison, nous affaiblissons les modèles pour le reste du document,
et les deux Chapitres~\ref{chapter:5} et \ref{chapter:6} étudient un modèle intermédiaire, situé entre le modèle MAB stationnaire à un joueur du chapitre~\ref{chapter:2} et les modèles de réseaux IdO du chapitre~\ref{chapter:4}.
\end{itemize}


% WARNING
\vfill{}

\textbf{Note sur le droit intellectuel.}
%
Ce document et les ressources additionnelles requises pour le compiler (notamment les fichiers \LaTeX, les morceaux de code Python, les figures etc)
sont \href{https://github.com/Naereen/phd-thesis/}{publiées publiquement},
selon les termes de la \href{https://lbesson.mit-license.org/}{\emph{licence MIT}} open-source,
en ligne sur \href{https://github.com/Naereen/phd-thesis/}{\texttt{GitHub.com/Naereen/phd-thesis/}}.

\TODOL{Open source the repository as soon as I defended my thesis!}

\begin{center}
    \textbf{Copyright 2016-2019, \copyright ~Lilian~Besson.}
\end{center}


\end{resume_fr}
