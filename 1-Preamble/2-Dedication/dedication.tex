% ******************************* Thesis Dedication ********************************

\begin{dedication}

    À mes très chers frères, mes deux plus grands modèles.

    À mes parents, mes deux plus grands soutiens.

    % À L., mon plus bel amour.

    \hr{}

    \vfill{}

    % https://fr.wikipedia.org/wiki/Liste_des_%C3%A9pisodes_de_Kaamelott

    \begin{small}
    \begin{quote}
        \emph{-- Je ne vous ai pas demandé d'en faire des \emph{bandits} non plus\dots}\\
        \emph{-- Des \emph{bandits}? Nooon\dots Des gars futés !}\\
        Alexandre Astier et Loïc Varraut,
        \emph{Kaamelott}, Livre II, Épisode 16 : ``Les tuteurs''.
        %  écrit par Alexandre Astier.

    % \hr{}
    \vspace*{4pt}

        \emph{-- Victoriae mundis et mundis lacrima. Bon, ça ne veut absolument rien dire, mais je trouve que c’est assez dans le ton.}\\
        François Rollin, \emph{Kaamelott}, Livre IV, Épisode 99, ``Le désordre et la nuit''.
        %  écrit par Alexandre Astier.

    % \hr{}
    \vspace*{4pt}

        \emph{
            % On devient pas chef parce qu'on le mérite andouille !
            % On devient chef par un concours de circonstances, on le mérite après !
            % Moi, il m'a p'têt fallu dix ans pour mériter mon grade, si pas vingt.
            % Tous les jours, j'ai travaillé pour pas nager dans mon uniforme.
            -- Y a pas trente-six solutions.
            Arturus ? Hein ? Fais semblant !
            % Fais semblant d'être Dux.
            Fais semblant de mériter ton grade.
            % Fais semblant d'être un grand chef de guerre.
            Si tu fais bien semblant, un jour tu verras, t'auras plus besoin !
        }\\
        Pierre Mondy, Kaamelott, Livre VI, Épisode 5, ``Dux Bellorum''.
        %  écrit par Alexandre Astier.

    % \hr{}
    \vspace*{4pt}

        \emph{-- Les rêves, ça se compare pas\ldots}\\
        Alexandre Astier, \emph{Kaamelott}, Livre VI, Épisode 9, ``Dies Irae''.
        %  écrit par Alexandre Astier.
    \end{quote}
    \end{small}

    %     Kaamelott saison 3 Episode 6
    % Léodagan: "Alors du coup, faut reconnaître on a quasiment plus de bandit. Par contre qu'est-ce qu'on a comme aveugle ! "


    % Kaamelott saison 6 Episode 6
    % Léodagan : "Moi je vous le dis, si on monte pas dans le char quand il nous passe sous le nez, on finira la route à pied."
    % Loth : "Oui, alors moi je pourrais vous dire que si on cueille pas les cerises quand elles sont sur l'arbre, on fera tintin pour le clafoutis. Mais on sera pas plus avancé."

    % https://fr.wikiquote.org/wiki/Kaamelott/Arthur


    % Je pense que vous glandouillez bien assez comme ça dans la réalité pour qu'on puisse se permettre d'optimiser le fictionnel.
    %
    % Alexandre Astier, Kaamelott, Livre IV, La Poétique II, écrit par Alexandre Astier.


    % Y'a trop de clampins qui se disent poètes qui sortent la licence poétique dès qu'ils pondent trois merdes que personne comprend .
    %
    % Alexandre Astier, Kaamelott, Livre II, Le Poème, écrit par Alexandre Astier.


    % C’était quand la dernière fois qu’on s’est retrouvés tous d’accord sur un truc !?
    %
    % Alexandre Astier, Kaamelott, Livre IV, Au service secret de Sa Majesté, écrit par Alexandre Astier.


    % Je suis le Roi Arthur, je ne désespère pas. Jamais je perds courage. Je suis un exemple pour les enfants.
    %
    % Alexandre Astier, Kaamelott, Livre VI, Dies Irae, écrit par Alexandre Astier.

    % \emph{-- Deus minimi placet : seul les dieux décident.}

    % François Rollin, \emph{Kaamelott}, Livre VI, ``Arturus Rex'', écrit par Alexandre Astier.

\end{dedication}