% ************************** Thesis Acknowledgements **************************

\begin{acknowledgements}

% See https://perso.crans.org/besson/phd/thanks/
\vspace*{-20pt}

I would like to acknowledge and thank the following people, and also the various fundings which supported our works during the last three years and made this thesis possible.

My PhD was founded by the French Ministry of Higher Education and Research (MENESR),
thanks to a ``contrat doctoral spécifique normalien'' obtained while I was a ``normalien'' student at \'Ecole Normale Sup\'erieure de Cachan (now ENS de Paris-Saclay).
This work has also been supported by
the French National Research Agency (ANR), under the projects BADASS, SOGREEN and EPHYL (grants \emph{N ANR-16-CE40-0002}, \emph{N ANR-14-CE28-0025-02} and \emph{N ANR-16-CE25-0002-03}),
by CNRS under the PEPS project BIO,
by European Union, through the European Regional Development Fund (ERDF),
% and by the Ministry of Higher Education and Research,
and by R\'egion Bretagne and Rennes Métropole, through the CPER Project \emph{SOPHIE / STIC \& Ondes}.

My thoughts then go to my two PhD advisors:\\
%
\indent
First, thanks to \href{https://MoyChristophe.Wordpress.com/}{Christophe Moy} who welcomed me in the SCEE team, in the IETR lab and in CentraleSupélec campus of Rennes, a nice and warm environment for research and day-to-day life.
Christophe was very supportive on every aspect of my work as a PhD student, and he was also helpful during difficult moments in the last three years. Many thanks for being so nice and dynamic,
% \footnote{~We had a nice first contact when we met in Spring 2015 in Mahindra Ecole Centrale, in Hyderabad, India, where I taught for one year, and I am honored to have pursued my PhD under his supervision and to have worked with him, and I am also happy to have lived great human moments alongside Christophe since 2016.}
and always comprehensive, especially regarding my teaching activities.
Christophe was helpful for his expertise on wireless communications and engineering, and especially on cognitive radio.
Regardless of what we did together, whether it was debugging a GNU Radio block \cite{Besson2018ICT,Besson2019WCNC}, writing a paper by night while Christophe was in a plane coming back from India \cite{MoyBesson2019}, chatting at a barbecue or playing ``palets bretons'' together along the ocean in Vannes, I was lucky to collaborate with and be around Christophe.\\
%
\indent
Then, thanks to \href{http://chercheurs.lille.inria.fr/ekaufman/research.html}{Émilie Kaufmann}, who accepted to co-supervise my thesis, and who did an amazing job in the last four years.
% \footnote{~Since our first discussions, in Autumn 2015 when she was giving practical sessions for the \href{http://researchers.lille.inria.fr/~lazaric/Webpage/MVA-RL_Course15.html}{Reinforcement Learning course} of \href{http://researchers.lille.inria.fr/~lazaric/}{Alessandro Lazaric} at MVA\footnote{~\href{http://math.ens-paris-saclay.fr/version-francaise/formations/master-mva/}{Master Mathématiques Vision Apprentissage} at \href{https://www.ens-paris-saclay.fr/}{ENS de Paris-Saclay}, previously named \href{https://www.ens-cachan.fr}{ENS de Cachan}.}, to the last repetitions of my PhD presentation in October 2019.}.
Émilie was highly motivated by our collaboration, and she always pushed me further, to explore the mathematical aspects of my PhD work, to keep faith in the research world, and to keep a strong rigour in every part of our exchanges and productions.
Visiting the SequeL team at Inria Lille on regular occasions was a rich experience, and I enjoyed working in close collaboration with Émilie during these visits.
% \footnote{~Merci à Romain et toi pour votre accueil chaleureux, et aussi merci à Florian, Nicolas et Édouard.}.
Our intensive white board sessions proved to be fruitful, giving the research work that I am the most proud of \cite{Besson2018ALT}, and rich research papers that join her rigorous expertise in statistical learning and mathematical analysis of bandit algorithms with my passion for numerical experiments, simulations and algorithm designs \cite{Besson2018ALT,Besson2018DoublingTricks,Besson2019GLRT}.

Then I would like to thank the dear members of my PhD committee.
It was interesting to be able to discuss with them on regular occasions, and these discussions have been very useful for some of our works.
Many thanks to \href{https://sites.google.com/site/vianneyperchet/}{Vianney Perchet} and \href{http://researchers.lille.inria.fr/~mitton/}{Nathalie Mitton} for accepting to review my thesis during the summer 2019, and for their very valuable comments.
Thanks to \href{http://perso.telecom-bretagne.eu/patrickmaille/}{Patrick Maillé} and \href{https://www.researchgate.net/profile/Raphael_Feraud/}{Raphaël Féraud} for following our work since Spring 2017,
and to \href{http://rcombes.supelec.free.fr/}{Richard Combes} as well for being part of my jury.
Some of their research works were strong inspirations for our own works, and I wish to also thank their co-authors, and in particular the Masters or PhD students, and useful discussions we had whenever we met: thanks to \href{https://www.cvernade.com/}{Claire Vernade}, \href{https://www.linkedin.com/in/reda-alami-067304109}{Réda Alami}, \href{https://www.researchgate.net/profile/Etienne_Boursier}{Étienne Boursier}, \href{https://www.researchgate.net/profile/Frederic_Loge_Munerel}{Frédéric Loge-Munerel} and \href{https://pratikgajane.wordpress.com/}{Pratik Gajane}, and others I could not meet (\eg, Viktor Toldov or \href{https://fr.linkedin.com/in/robin-allesiardo-phd-39221a92}{Robin Allesiardo}).

Of course on a more personal level, my thoughts are strongly dedicated to my family.
%
I first want to thank my \href{https://paris-sorbonne.academia.edu/FBesson}{older brother Florian}, who has always been my main model and who has been a constant support during all my life and in my studies.
His rigour, dedication and seriousness kept pushing me further, and I am immensely proud of him.
Some of our board game nights were brain teasers comparable with some maths problems solved for this thesis!
% and I was happy to help him for the organization of the \href{http://lettres.sorbonne-universite.fr/Kaamelott}{research colloque}\footnote{~Un livre a aussi été publié suite \href{http://lettres.sorbonne-universite.fr/Kaamelott}{au colloque}, \href{https://www.editions-vendemiaire.com/catalogue/a-paraitre/kaamelott-un-livre-d-histoire-florian-besson-et-justine-breton-dir/}{``Kaamelott, un livre d'histoire'', aux éditions Vandémiaire}.} on the \emph{Kaamelott} french TV show in February 2017, it was \href{https://www.fabula.org/actualites/kaamelott-ou-la-relecture-de-l-histoire_75490.php}{the funniest} and \href{https://www.vousnousils.fr/2017/03/22/kaamelott-un-tremplin-vers-lhistoire-et-la-litterature-601428}{most exciting} \href{https://lpcm.hypotheses.org/10718}{research event} I have been to in the last four years!
%
I also thank my younger brother Fabian, for being an inspiration for other (highly important) directions of my life. \emph{Tes voyages, ton courage et ta force m'inspirent quotidiennement, merci beaucoup !}
I sincerely thank my parents C. \& P., for being the best parents since the last $26$ years, and for their (vital) moral support.
\emph{Merci infiniment de nous avoir offert cette curiosité, merci d'être si gentils, et merci de votre soutien constant et de votre amour !}
Very warm thanks then go to M., to my cousin L. and my grand-parents, J. \& J.-M. and J., to my uncle J. and aunt B.!
\emph{Merci enfin à M., N. \& B. pour votre accueil à La Gouesnière !}
% \\
% \indent

I would then like to thank my closest friends, sorted in chronological order of our first encounter in the \href{https://bulbapedia.bulbagarden.net/wiki/Tall_grass}{tall grass}\footnote{~Sorry if I forgot you, it's harder to keep track of wild encounters in the tall grass without a Pokédex!}: Guillaume, Florian, Marian, Hélène, Lucie, Lætitia and Julia in Briançon, Mayotte and Pierre in Marseille, Laurent, Simon, Tristan, Ludovic, Jessica, Romain, Benjamin, Angèle, Clément in Cachan, Jill-Jênn, Alain, Claire, Édouard, Loïc and Damien also in Cachan, Kumudham and Priscilla in India, Thibault, Sébastien, Sélim, Valentin again in Cachan, and Audrey, Rémi, Marine, Adrien, Bastien, Claire  and Lola in Rennes.
%  And due to space constraints, I had to choose quite arbitrarily whom to include\dots
% In the last four years, thank you so much my dear friends, for great moments and an even greater moral support, for the holidays in India, Switzerland, Hungary/Austria, Strasbourg, Greece, Saint-Malo, Normandy, Scotland, Marseille, Madrid, Brittany etc. I loved traveling with you and hope to continue!
\emph{J'ai une pensée particulière pour Hélène, merci pour ta foi, ton énergie et ta détermination, notamment en 2018-19.}

Thanks to my wonderful colleagues during these three years: Rémi\textsuperscript{\tiny{\Coffeecup}}, Pascal\textsuperscript{\tiny{\Coffeecup}}, Yves, Amor, Carlos Faouzi, Jacques P., Jacques W., Ali C., Haïfa, Marwa, Navik, Quentin\textsuperscript{\tiny{\Coffeecup}}, Vincent\textsuperscript{\tiny{\Coffeecup}}, Rami\textsuperscript{\tiny{\Coffeecup}}, Pierre H., Muhammad, Cristo\textsuperscript{\tiny{\Coffeecup}}, Adrien\textsuperscript{\tiny{\Coffeecup}}, Esteban\textsuperscript{\tiny{\Coffeecup}}, Ali Z., Éloïse\textsuperscript{\tiny{\Coffeecup}}, Nabil\textsuperscript{\tiny{\Coffeecup}}, Bastien\textsuperscript{\tiny{\Coffeecup}}, Corentin\textsuperscript{\tiny{\Coffeecup}}, Julio, Georgios, Morgane\textsuperscript{\tiny{\Coffeecup}} in Rennes ; and Alessandro, Ronan, Julien P., Odalric-Ambrym, Philippe, Olivier, Florian, Xuedong, Guillaume, Benjamin, Julien S., Mathieu, Matteo, Mahsa, Nicolas, Sadegh, Édouard, Yannis and Omar in Lille.
I want to especially thank the ``coffee friends\textsuperscript{\tiny{\Coffeecup}}'' in Rennes, for our daily interesting discussions.
%  the best moments of our life at the office.
%
Thanks to the \emph{``futsal''} players on Wednesday lunch,
% I still have to progress immensely,
% before I could pretend to be useful
I haven't been useful, but it's been a lot of fun to play with you guys!


In the last few years, some friends thanked me in their own PhD thesis, and it is a pleasure to do the same, and cross-link to their own PhD thesis.
In a chronological order, I thank
%,
\href{http://jill-jenn.net/}{Jill-Jênn Vie} (\href{https://github.com/jilljenn/phd}{thesis link $\nearrow$}), \href{https://navikkumarmodi.wordpress.com/}{Navikkumar Modi} (\href{https://tel.archives-ouvertes.fr/tel-01668536/document}{$\nearrow$}), \href{https://quentinbodinier.wordpress.com/}{Quentin Bodinier} (\href{https://tel.archives-ouvertes.fr/tel-01731022/document}{$\nearrow$}), \href{https://paris-sorbonne.academia.edu/FBesson}{Florian Besson}\footnote{~Que je remercie deux fois plus que ce qu'il me remerciait dans sa propre thèse ! Pour répondre à la question, deux ans plus tard, tu remerciais Tom et moi un nombre non nul $x\in\N\cup\{+\infty\}$ satisfaisant $x=2x$, soit $x=+\infty$, et je t'en remercie $2x$ fois !}, \href{https://sites.google.com/view/guerand}{Jessica Guérand} (\href{https://www.theses.fr/s175725}{the}\href{https://www.theses.fr/s175725}{$\nearrow$}), \href{https://sites.google.com/view/ducasse/}{Romain Ducasse} (\href{https://drive.google.com/open?id=1u2oxRRimcO0jQfuYwSVwgKfcHU5DdoPK}{$\nearrow$}), \href{https://members.loria.fr/SAbelard/}{Simon Abelard} (\href{https://members.loria.fr/SAbelard/theseabelard.pdf}{$\nearrow$}), \href{http://www.cmap.polytechnique.fr/~sacchelli/}{Ludovic Sacchelli} (\href{https://tel.archives-ouvertes.fr/tel-01893068/document}{$\nearrow$}), \href{http://pages.saclay.inria.fr/claire.brecheteau/}{Claire Brécheteau} (\href{https://hal.archives-ouvertes.fr/tel-01897787/document}{$\nearrow$}), \href{http://www.i2m.univ-amu.fr/perso/damien.allonsius/}{Damien Allonsius} (\href{http://www.i2m.univ-amu.fr/perso/damien.allonsius/documents/recherche/these/Main.pdf}{$\nearrow$}), \href{https://remibonnefoi.wordpress.com/}{Rémi Bonnefoi}, and Rami Othman.


For their administrative support regarding research, travelling or teaching,
thanks to Karine, Jeannine, Grégory, Anne, Cécile, Maud, Frédéric, Gabriel (and others) at CentraleSupélec campus of Rennes, Amélie at Inria Lille and Maryline (and others) at CRIStAL in Lille.
Thanks to the nice people at Sodexo in our campus, especially to Martine and J.-B.!
%  your daily smile was priceless!

During my PhD, I was lucky to teach regularly in Mathematics and Computer Science.
I warmly thank David Pichardie for his trust and support, and also François Schwarzentruber, Nathalie Bertrand and David Cachera at ENS de Rennes, and Romaric Gaudel and his colleagues at ENSAI. Thanks to them, I loved to teach these $3 \times 64$ hours.
% since October $2016$.
% I am trully honored to join \emph{the Département d'Informatique} at ENS de Rennes in next September $2019$.
Thanks to my amazing students, I am proud of you and of our interesting exchanges.
%  and I feel lucky to be able to keep working alongside some of you.
I am thankful to Amélie Stainer and Gilbert Gabillard at Lycée Chateaubriand and Joliot-Curie for useful discussions.
\\
\indent
I am grateful to amazing professors who passed on their passion for mathematics and computer science to me.
A partial list includes Hubert P. in Briançon ($2006$-$09$), Yassine D. ($2009$-$10$) and Agnès B. ($2010$-$11$) in Lycée Thiers in Marseille.
Then at ENS de Cachan ($2011$-$16$), I especially think about Frédéric P., Alain T., Claudine P., Florian de V. and Nicolas V. in Maths, to Hubert C.-L., Jean G.-L., Sylvain S., Paul G., Serge H. in Computer Science, and to Carine S.-P., Delphine L. and Nicolas P.\textsuperscript{$\dagger$} for their support.
Thanks to Didier C. and Vijaysekhar C. in Hyderabad ($2014$-$15$), Michael U. and Julien F. in Lausanne ($2016$) for our collaborations.


I would also like to thank some unusual things.
I thank my \href{https://perso.crans.org/besson/zero-dechet/}{bike} which I enjoyed riding everyday, despite the too many accidents and frequent rain, my \href{https://perso.crans.org/besson/cuisine/}{enthusiastic daily cooking} that gave me smiles and energy, and my \href{https://perso.crans.org/besson/}{favorite open-soure technological tools}: git (and GitHub and Bitbucket), \LaTeX, Bash, Julia and Python, GNU/Linux and (X)Ubuntu, Firefox, Konsole and Visual Studio Code, and many other great free and open-source tools.
Some people thank (a/some) God in their thesis, I feel that I prefer to thank \emph{the Earth}, and apologize for the too many times I took a plane for a conference abroad, and in general apologize for the environmental impact of this research work. I am trying hard to do better from now on!


\vfill{}

\emph{Je suis désolé si je vous ai oublié, merci de corriger} : Je remercie sincèrement ....................... pour .............
\emph{I am sorry I forgot you, please fix this mistake}: I warmly thank ..................... for ...............

\vfill{}

% I would like to conclude these acknowledgements by a special thought toward Lola, who was of great support on a daily basis, who marvelled me by her enthusiasm and dynamism, and who comforted me with her laughs and love, in the last year.
% \textcolor{white}{I love you Lola, would you like to live with me?}
%
% Je souhaite terminer ces remerciements par une pensée toute particulière pour Lola, qui m'a soutenu quotidiennement, qui m'a émerveillé par son enthousiasme et qui m'a réconforté par son amour, durant la dernière année.
% \textcolor{white}{Je t'aime Lola, veut-tu vivre avec moi ?}
%
% \small{
    \emph{Deseo terminar este agradecimiento con un pensamiento especial para mi gatita \emph{L.}, quien me apoyó diariamente, me sorprendió con su entusiasmo y me consoló con sus risas y su amor, durante el año pasado. ¡Muchísimas gracias, gatita!}
    \textcolor{white}{$\heartsuit$ Te quiero, Lola, ¿Quieres compartir tu vida conmigo?}
% }

\end{acknowledgements}

