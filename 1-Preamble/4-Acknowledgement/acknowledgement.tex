% ************************** Thesis Acknowledgements **************************

\begin{acknowledgements}

% \TODOL{FIXME à finir, en septembre !}
% See https://perso.crans.org/besson/phd/thanks/

I would like to acknowledge and thank the following people, and also acknowledge the various fundings who supported our works during the last three years and made this thesis possible.

My first thoughts go to my two PhD advisors.\\
%
\indent
Firstly, thanks to \href{https://MoyChristophe.Wordpress.com/}{Christophe Moy} who welcomed me in the SCEE team, in IETR lab and in CentraleSupélec campus of Rennes, a very nice and warm environment for research and day-to-day life.
Christophe was very supportive on all aspects of the PhD work, and he was also helpful during difficult moments in the last three years. Many thanks for being so nice and dynamic\footnote{~We had a nice first contact when we met in Spring 2015 in Mahindra Ecole Centrale, in Hyderabad, India, where I taught for one year, and I am honored to have pursued my PhD under his supervision and to have worked with him, and I am also happy to have lived great human moments alongside Christophe since 2016.}, and always comprehensive.
Christophe was helpful for his expertise on wireless communications and engineering, and especially on cognitive radio.
Regardless of what we did together, whether it was debugging a difficult GNU Radio block \cite{Besson2018ICT,Besson2019WCNC}, writing a paper by night while Christophe was in a plane coming back from India \cite{MoyBesson2019}, chatting at a barbecue or playing ``palets'' together along the ocean in Vannes, I was lucky to collaborate with and be around Christophe.\\
%
\indent
Secondly, thanks to \href{http://chercheurs.lille.inria.fr/ekaufman/research.html}{Émilie Kaufmann}, who accepted to co-supervised my thesis, and did an amazing job in the last four years\footnote{~Since our first discussions, in Autumn 2015 when she was giving practical sessions for the \href{http://researchers.lille.inria.fr/~lazaric/Webpage/MVA-RL_Course15.html}{Reinforcement Learning course} of \href{http://researchers.lille.inria.fr/~lazaric/}{Alessandro Lazaric} at MVA\footnote{~\href{http://math.ens-paris-saclay.fr/version-francaise/formations/master-mva/}{Master Mathématiques Vision Apprentissage} at \href{https://www.ens-paris-saclay.fr/}{ENS de Paris-Saclay}, previously named \href{https://www.ens-cachan.fr}{ENS de Cachan}.}, to the last repetitions of my PhD presentation in October 2019.}.
Émilie was highly motivated by our collaboration, and she always pushed me further, to explore the mathematical aspects of my PhD work, to keep faith in the research world, and to keep the highest possible rigour in all parts of our exchanges and productions.
I was lucky to be able to visit the SequeL team at Inria Lille on regular occasions, and to work in close collaboration with Émilie during these visits\footnote{~Merci à Romain et toi pour votre accueil chaleureux, et aussi merci à Florian, Nicolas et Édouard.}. Intensive white board sessions proved to be fruitful, giving the research work that I am the most proud of \cite{Besson2018ALT}, and rich research papers that join her rigorous expertise in statistical learning and mathematical analysis of bandit algorithms with my passion for numerical experiments, simulations and algorithm designs \cite{Besson2018ALT,Besson2018DoublingTricks,Besson2019GLRT}.

Then I would like to thank the dear members of my PhD committee.
I was lucky to be able to discuss with them on regular occasions, and these discussions have been very useful for some of our works.
Many thanks to \href{https://sites.google.com/site/vianneyperchet/}{Vianney Perchet} and \href{http://researchers.lille.inria.fr/~mitton/}{Nathalie Mitton} for accepting to review my thesis during the summer 2019, and for their very valuable comments.
Thanks to \href{http://perso.telecom-bretagne.eu/patrickmaille/}{Patrick Maillé} and \href{https://www.researchgate.net/profile/Raphael_Feraud/}{Raphaël Féraud} for following our work since Spring 2017,
and to \href{http://rcombes.supelec.free.fr/}{Richard Combes} as well for being part of my jury.
Some of their articles were strong inspirations for our own works, and I wish to also thank their co-authors, and in particular the Masters or PhD students, and useful discussions we had whenever we met, especially thanks to \href{https://www.cvernade.com/}{Claire Vernade}, \href{https://www.linkedin.com/in/reda-alami-067304109}{Réda Alami}, \href{https://www.researchgate.net/profile/Etienne_Boursier}{Étienne Boursier}, \href{https://www.researchgate.net/profile/Frederic_Loge_Munerel}{Frédéric Loge-Munerel} and \href{https://pratikgajane.wordpress.com/}{Pratik Gajane}, and others I could not meet (\eg, Viktor Toldov or \href{https://fr.linkedin.com/in/robin-allesiardo-phd-39221a92}{Robin Allesiardo}).

Of course on a more personal level, my thoughts are strongly dedicated to my family.
%
I first want to thank my \href{https://paris-sorbonne.academia.edu/FBesson}{older brother Florian}, who has always been my main model and who has been of a constant support during all my studies. His solid and serious work kept pushing me further, some of our board games nights were brain teasers comparable with some maths problems solved for this thesis.
% and I was happy to help him for the organization of the \href{http://lettres.sorbonne-universite.fr/Kaamelott}{research colloque}\footnote{~Un livre a aussi été publié suite \href{http://lettres.sorbonne-universite.fr/Kaamelott}{au colloque}, \href{https://www.editions-vendemiaire.com/catalogue/a-paraitre/kaamelott-un-livre-d-histoire-florian-besson-et-justine-breton-dir/}{``Kaamelott, un livre d'histoire'', aux éditions Vandémiaire}.} on the \emph{Kaamelott} french TV show in February 2017, it was \href{https://www.fabula.org/actualites/kaamelott-ou-la-relecture-de-l-histoire_75490.php}{the funniest} and \href{https://www.vousnousils.fr/2017/03/22/kaamelott-un-tremplin-vers-lhistoire-et-la-litterature-601428}{most exciting} \href{https://lpcm.hypotheses.org/10718}{research event} I have been to in the last four years!
%
I also thank my younger brother Fabian, for being an inspiration for other (highly important) directions of my life.
Very warm thanks then go to Marine, to my grand-parents, Jeannine, Jean-Marc and Jacqueline, my uncle Jacques and aunt Bernadette, and my cousins.
Finally, I thank sincerely my parents, Christophe and Patricia, for being the best parents since the last 30 years, and for their vital moral support. \emph{Merci de nous avoir inculqué une si forte curiosité et d'être si gentils !}

I would like to thank my closest friends, sorted in chronological order of first encounter in the tall grass\footnote{~Sorry if I forgot you, it's much harder to keep track of wild encounters in the tall grass in real wife, without a Pokédex! And due to space constraints, I had to chose quite arbitrarily who to include\dots}: Florian, Fabian, Guillaume, Florian, Marian, Hélène, Lucie, Lætitia and Julia from Briançon, Mayotte and Pierre from Marseille, Laurent, Simon, Tristan, Ludovic, Jessica, Romain, Benjamin, Angèle, Clément from Cachan, Jill-Jênn, Alain, Claire, Édouard, Loïc and Damien also from Cachan, Kumudham and Priscilla from India, Thibault, Sébastien, Sélim, Valentin again from Cachan, and Rémi, Marine, Adrien, Bastien, Claire, Corentin and Lola from Rennes.
% In the last four years, thank you so much my dear friends, for great moments and an even greater moral support, for the holidays in India, Switzerland, Hungary/Austria, Strasbourg, Greece, Saint-Malo, Normandy, Scotland, Marseille, Madrid, Brittany etc. I loved traveling with you and hope to continue!

Thanks to my wonderful colleagues during these three years: Rémi$^*$, Pascal$^*$, Yves, Jacques, Ali, Haïfa, Marwa, Navik, Quentin$^*$, Vincent$^*$, Rami$^*$, Muhammad, Cristo$^*$, Adrien$^*$, Esteban$^*$, Ali, Éloïse$^*$, Nabil$^*$, Bastien$^*$, Corentin$^*$, Georgios, Morgane$^*$ in Rennes ; and Ronan, Odalric-Ambrym, Philippe, Olivier, Florian, Xuedong, Guillaume, Julien, Mathieu, Matteo, Mahsa, Nicolas, Edouard in Lille.
I want to thank especially the ``coffee friends$^*$'' in Rennes, for the daily interesting discussions, which where sometimes the highlights of our life at the office.
%
Thanks to everybody who played \emph{futsal} on Wednesday lunch, I still have to progress immensely before I could pretend to be useful, but it's been a lot of fun to play with you guys!

Thanks to some people for their administrative support regarding research, travelling or teaching: Karine, Jeannine, Grégory, Anne, Cécile (and others) at CentraleSupélec campus of Rennes, Amélie at Inria Lille and Maryline (and others) at CRIStAL in Lille.
Thanks to the nice people at Sodexo in the CentraleSupélec campus of Rennes, especially Martine and J-B, your daily smile is priceless!
I warmly thank David P. for his trust, and François, David P. and David C. at ENS de Rennes, and Romaric at ENSAI : I loved every single of the $3 \times 64$ hours I spent teaching with you in the last three years.
I am grateful to Amélie at Lycée Chateaubriand and Gilbert at Lycée Joliot-Curie for useful exchanges.
%
I am thankful to all the great professors who taught me during my studies at ENS de Cachan, between 2011 and 2016, and for some who pushed me into teaching, and studying statistical learning and computer science, including Frédéric, Alain and Claudine in the maths department, Sylvain, Paul, Serge in the computer science department, and thanks to Carine, Delphine and the late Nicolas for their support.


% TODO financial thanks
This work and my PhD is supported by
the French National Research Agency (ANR), under the projects BADASS, SOGREEN and EPHYL (grants \emph{N ANR-16-CE40-0002}, \emph{N ANR-14-CE28-0025-02} and \emph{N ANR-16-CE25-0002-03}),
by R\'egion Bretagne, France,
by CNRS under the PEPS project BIO (in 2017),
by the French Ministry of Higher Education and Research (MENESR),
by \'Ecole Normale Sup\'erieure de Paris-Saclay.
by European Union, through the European Regional Development Fund (ERDF),
and by Ministry of Higher Education and Research, Brittany and Rennes Métropole, through the CPER Project \emph{SOPHIE / STIC \& Ondes}.
% I was also lucky to be well received very frequently at Inria Lille Nord Europe, and to be hosted at CentraleSupélec campus of Rennes.


I would also like to thank some unusual things. Some people thank God in their thesis, I feel that I prefer to thank the Earth, and apologize for the too many times I took a plane for a conference abroad, and in general apologize for the environmental impact of this research work. I'll try hard to do better from now on!
I also thank my \href{https://perso.crans.org/besson/zero-dechet/}{bike} which I enjoyed riding everyday despite the too many accidents and rain, my \href{https://perso.crans.org/besson/cuisine/}{enthusiastic daily cooking} who gave me energy (\eg, cheese, chocolate, coffee, vegetables etc), and finally my \href{https://perso.crans.org/besson/}{favorite technological tools}: git, GitHub and Bitbucket, \LaTeX, Bash and Python, GNU/Linux and XUbuntu, Firefox, Konsole and Visual Studio Code, and many other free and open-source tools.


In the last few years, some friends thanked me in their own PhD thesis, and it is a pleasure to do the same, and cross-link to their own PhD thesis.
In a chronological order, I thank
%,
\href{http://jill-jenn.net/}{Jill-Jênn Vie} (\href{https://github.com/jilljenn/phd}{thesis}), \href{https://navikkumarmodi.wordpress.com/}{Navikkumar Modi} (\href{https://tel.archives-ouvertes.fr/tel-01668536/document}{thesis}), \href{https://quentinbodinier.wordpress.com/}{Quentin Bodinier} (\href{https://tel.archives-ouvertes.fr/tel-01731022/document}{thesis}), \href{https://paris-sorbonne.academia.edu/FBesson}{Florian Besson}\footnote{~Que je remercie deux fois plus que ce qu'il me remerciant dans sa propre thèse ! Pour répondre à la question, deux ans plus tard, tu remerciais Tom et moi un nombre non nul $x\in\N\cup\{+\infty\}$ satisfaisant $x=2x$, soit $x=+\infty$, et je t'en remercie $2x$ fois !}, \href{https://sites.google.com/view/guerand}{Jessica Guérand} (\href{https://www.theses.fr/s175725}{the}\href{https://www.theses.fr/s175725}{sis}), \href{https://sites.google.com/view/ducasse/}{Romain Ducasse} (\href{https://drive.google.com/open?id=1u2oxRRimcO0jQfuYwSVwgKfcHU5DdoPK}{thesis}), \href{https://members.loria.fr/SAbelard/}{Simon Abelard} (\href{https://members.loria.fr/SAbelard/theseabelard.pdf}{thesis}), \href{http://www.cmap.polytechnique.fr/~sacchelli/}{Ludovic Sacchelli} (\href{https://tel.archives-ouvertes.fr/tel-01893068/document}{thesis}), \href{http://pages.saclay.inria.fr/claire.brecheteau/}{Claire Brécheteau} (\href{https://hal.archives-ouvertes.fr/tel-01897787/document}{thesis}), \href{http://www.i2m.univ-amu.fr/perso/damien.allonsius/}{Damien Allonsius} (\href{http://www.i2m.univ-amu.fr/perso/damien.allonsius/documents/recherche/these/Main.pdf}{thesis}), \href{https://remibonnefoi.wordpress.com/}{Rémi Bonnefoi}, Rami Othman.

% I would like to conclude these acknowledgements by a special thought toward Lola, who was of great support on a daily basis, who marvelled me by her enthusiasm and dynamism, and who comforted me with her laughs and love, in the last year.
% \textcolor{white}{I love you Lola, would you like to live with me?}
%
% Je souhaite terminer ces remerciements par une pensée toute particulière pour Lola, qui m'a soutenu quotidiennement, qui m'a émerveillé par son enthousiasme et qui m'a réconforté par son amour, durant la dernière année.
% \textcolor{white}{Je t'aime Lola, veut-tu vivre avec moi ?}
%
\emph{Deseo terminar este agradecimiento con un pensamiento especial para Lola, quien me apoyó diariamente, me sorprendió con su entusiasmo y me consoló con sus risas y su amor, durante el año pasado.}\\
\indent
\textcolor{white}{$\heartsuit$ Te quiero, Lola, ¿quieres vivir conmigo?}

\end{acknowledgements}
