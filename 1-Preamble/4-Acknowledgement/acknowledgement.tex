% ************************** Thesis Acknowledgements **************************

\begin{acknowledgements}

\TODOE{FIXME à finir, en septembre !}
% See https://perso.crans.org/besson/phd/thanks/

I would like to acknowledge and thank the following people, and also acknowledge the various funding who supported our works during the last three years.

My first thoughts go to my two PhD advisors.
%
Firstly, thanks to \href{https://MoyChristophe.Wordpress.com/}{Christophe Moy} who welcomed me in the SCEE team, in the IETR lab and in CentraleSupélec campus of Rennes, a very nice and warm environment not only for research but also for day-to-day life.
Christophe was very supportive, on all aspects of the PhD work and he was also helpful during difficult moments in the last three years. Many thanks for being so nice and dynamic, and always comprehensive.
Christophe was immensely helpful for his expertise on wireless communications and engineering, and especially on cognitive radio.
Regardless of what we did together, debugging a difficult GNU Radio block, writing a paper together by night in a plane coming back from India, chatting around a friendly barbecue or playing ``palets'' together along the ocean in Vannes, I was extremely lucky to collaborate with and be around Christophe.
We had a nice first contact when we met in Spring 2015 in Mahindra Ecole Centrale, in Hyderabad, India, where I taught for one year, and I am honored to have pursued my PhD under his supervision and to have worked with him, and I am also happy to have lived great human moments along with Christophe since 2016.
%
Secondly, thanks to \href{http://chercheurs.lille.inria.fr/ekaufman/research.html}{Émilie Kaufmann}, who accepted to co-supervised my thesis, and did an amazing job in the last four years. Since our first discussions, in Autumn 2015 when she was giving practical sessions for the \href{http://researchers.lille.inria.fr/~lazaric/Webpage/MVA-RL_Course15.html}{Reinforcement Learning course} of \href{http://researchers.lille.inria.fr/~lazaric/}{Alessandro Lazaric} at MVA\footnote{\href{http://math.ens-paris-saclay.fr/version-francaise/formations/master-mva/}{Master Mathématiques Vision Apprentissage} at \href{https://www.ens-paris-saclay.fr/}{ENS de Paris-Saclay}, previously named \href{https://www.ens-cachan.fr}{ENS de Cachan}.}, to the last repetitions of my PhD presentation in October 2019.
Emilie was highly motivated by our collaboration, and she always pushed me further, to explore the mathematical aspects of my PhD work, to keep faith in the research world, and to keep the highest possible rigour in all parts of our exchanges.
I was lucky to be able to visit the SequeL team at Inria Lille on regular occasions, and to work in close collaboration with Emilie during these visits. Intensive white board sessions proved to be fruitful, giving the research work that I am the most proud of \cite{Besson2018ALT}, and rich research papers that join her rigorous expertise in statistical learning and mathematical analysis of bandit algorithms with my passion for numerical experiments, simulations and algorithm designs \cite{Besson2018ALT,Besson2018DoublingTricks,Besson2019GLRT}.

Then I would like to thank the dear member of my PhD jury.
Many thanks to \href{https://sites.google.com/site/vianneyperchet/}{Vianney Perchet} and \href{http://researchers.lille.inria.fr/~mitton/}{Nathalie Mitton} for accepting to review my thesis during the summer 2019, and for their very valuable comments.
Thanks to Patrick Maillé and Raphaël Féraud for following our work since Spring 2017,
and to Richard Combes as well for being part of my jury.

Of course, my thought then go to my family: my parents, Christophe and Patricia, my \href{https://paris-sorbonne.academia.edu/FBesson}{older brother Florian} and my younger brother Fabian, Marine, my grand-parents, Jeannine, Jean-Marc and Jacqueline, my uncle and aunt Jacques, Bernadette, and my cousin Lara.
\TODOL{Une belle phrase sur leur soutien ? En particulier les parents et l'éducatin qu'ils nous ont donnés, Florian et son soutien inconditionnel et son rôle de modèle, Fabian pour ?}

I would like to thank my closest friends, sorted in chronological order of first encounter in the tall grass\footnote{Sorry if I forgot you, }: Florian, Fabian, Guillaume, Florian, Marian, Hélène, Lucie, Lætitia and Julia from Briançon, Mayotte and Pierre from Marseille, Laurent, Simon, Tristan, Ludovic, Jessica, Romain, Benjamin, Angèle, Clément, Jill-Jênn, Alain, Claire, Édouard, Loïc and Damien from Cachan, Kumudham and Priscilla from India, Thibault, Sébastien, Sélim, Valentin from Cachan again, and Rémi, Marine, Adrien, Bastien, Claire, Corentin and Lola from Rennes.
In the last four years, thank you so much my dear friends, for great moments and an even greater moral support, for the holidays in Hungary/Ausria, Strasbourg, Greece, Saint-Malo, Normandy, Scotland, Marseille, Madrid, Nantes\dots I loved traveling with you and hope to continue!

Thanks to my wonderful colleagues during these three years: Rémi, Pascal, Yves, Jacques, Ali, Haïfa, Marwa, Navik, Quentin, Vincent, Rami, Muhammad, Cristo, Adrien, Esteban, Ali, Éloïse, Nabil, Bastien, Corentin, Georgios, Morgane in Rennes ; and Ronan, Odalric-Ambrym, Philippe, Olivier, Florian, Xuedong, Guillaume, Julien, Mathieu, Matteo, Mahsa, Nicolas in Lille.
%
Thanks to some people for their administrative support regarding research, travelling or teaching: Karine, Jeannine, Grégory, Anne, Cécile and others at CentraleSupélec campus of Rennes; François, David, David at ENS de Rennes, Romaric at ENSAI, Amélie at Inria Lille, Amélie at Lycée Chateaubriand, and Gilbert at Lycée Joliot-Curie.
Thanks to the nice people at Sodexo in the CentraleSupélec campus of Rennes, especially Martine and J-B, your daily smile is priceless.
%
During my studies at ENS de Cachan, I am thankful to all the great professors who taught me and pushed me into studying statistical learning and computer science, including Frédéric, Alain and Claudine in the maths department, Sylvain, Paul, Serge in the computer science department, and thanks to Carine, Delphine and the late Nicolas for their support.


% TODO financial thanks
This work is supported by
the French National Research Agency (ANR), under the projects BADASS, SOGREEN and EPHYL (grants \emph{N ANR-16-CE40-0002}, \emph{N ANR-14-CE28-0025-02} and \emph{N ANR-16-CE25-0002-03}),
by R\'egion Bretagne, France,
by CNRS under the PEPS project BIO (in $2017$),
by the French Ministry of Higher Education and Research (MENESR),
by \'Ecole Normale Sup\'erieure de Paris-Saclay.
by European Union, through the European Regional Development Fund (ERDF),
and by Ministry of Higher Education and Research, Brittany and Rennes Métropole, through the CPER Project \emph{SOPHIE / STIC \& Ondes}.


I would also like to thank some things. First, the Earth, then my \href{https://perso.crans.org/besson/zero-dechet/}{bike} which I enjoyed riding everyday despite the too many accidents.
My \href{https://perso.crans.org/besson/cuisine/}{culinary friends}: cheese, chocolate, coffee, vegetables etc!
My \href{https://perso.crans.org/besson/}{technological friends}: Bash, Python, XUbuntu, Firefox, Konsole, Visual Studio Code etc.


Some friends thanked me in their own PhD thesis, and it is a pleasure to do the same, and cross-link to their own PhD thesis.
In a chronological order, I thank
%,
\href{http://jill-jenn.net/}{Jill-Jênn Vie} (\href{http://jiji.cat/cat.pdf}{thesis}), \href{https://navikkumarmodi.wordpress.com/}{Navikkumar Modi} (\href{https://tel.archives-ouvertes.fr/tel-01668536/document}{thesis}), \href{https://quentinbodinier.wordpress.com/}{Quentin Bodinier} (\href{https://tel.archives-ouvertes.fr/tel-01731022/document}{thesis}), \href{https://paris-sorbonne.academia.edu/FBesson}{Florian Besson}, \href{https://sites.google.com/view/guerand}{Jessica Guérand} (\href{https://www.theses.fr/s175725}{the}\href{https://www.theses.fr/s175725}{sis}), \href{https://sites.google.com/view/ducasse/}{Romain Ducasse} (\href{https://drive.google.com/open?id=1u2oxRRimcO0jQfuYwSVwgKfcHU5DdoPK}{thesis}), \href{https://members.loria.fr/SAbelard/}{Simon Abelard} (\href{https://members.loria.fr/SAbelard/theseabelard.pdf}{thesis}), \href{http://www.cmap.polytechnique.fr/~sacchelli/}{Ludovic Sacchelli} (\href{https://tel.archives-ouvertes.fr/tel-01893068/document}{thesis}), \href{http://pages.saclay.inria.fr/claire.brecheteau/}{Claire Brécheteau} (\href{https://hal.archives-ouvertes.fr/tel-01897787/document}{thesis}), \href{http://www.i2m.univ-amu.fr/perso/damien.allonsius/}{Damien Allonsius} (\href{http://www.i2m.univ-amu.fr/perso/damien.allonsius/documents/recherche/these/Main.pdf}{thesis}), \href{https://remibonnefoi.wordpress.com/}{Rémi Bonnefoi}, Rami Othman.

% Je souhaite terminer ces remerciements par une pensée toute particulière pour Lola, qui m'a soutenu quotidiennement, qui m'a émerveillé par son enthousiasme et qui m'a réconforté par son amour, durant la dernière année. Je t'aime Lola, veut-tu vivre avec moi ?
Deseo terminar este agradecimiento con un pensamiento especial para Lola, quien me apoyó diariamente, me sorprendió con su entusiasmo y me consoló con su amor, durante el año pasado.
\textcolor{white}{Je t'aime Lola, veux-tu vivre avec moi ?}

\end{acknowledgements}
